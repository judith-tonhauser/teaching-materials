
\documentclass[12pt,fleqn,a4]{extarticle}
\usepackage{longtable}
\usepackage{enumitem}

\usepackage[dvips]{graphics}
\usepackage{hyperref}
\usepackage{txfonts}
\usepackage{tipa}
\usepackage{amsfonts}
\usepackage{natbib}
\usepackage{gb4e}
\usepackage[all]{xy}

\setlength{\textwidth}{7in}
\setlength{\textheight}{9in}
\setlength{\topmargin}{0in} %has to be 1.2 inch
\setlength{\oddsidemargin}{-.3in}
%\setlength{\evensidemargin}{0in}
\setlength{\parindent}{0in}
\setlength{\parskip}{1ex}
\setlength{\headsep}{0in}

\setlength{\bibsep}{0mm}
\bibpunct[:]{(}{)}{;}{a}{,}{,}

\newcommand{\yi}{\'{\symbol{16}}}
\newcommand{\nasi}{\~{\symbol{16}}}
\newcommand{\hina}{h\nasi na}
\newcommand{\ina}{\nasi na}
%\renewcommand{\abut}{$\supset$\hspace*{-0.07cm}$\subset$}
\newcommand{\tto}{t$_{top}$}
\newcommand{\wtop}{w$_{top}$}
\newcommand{\tc}{t$_c$}
\newcommand{\schwa}{\begin{sideways}e\end{sideways}}

% Semantic brackets
%\newcommand{\iss}[1]{\mbox{\protect\tiny \mbox{#1}}}
%\newcommand{\sem}[2]{\6#1\9$_\iss{#2}$} David's original
\newcommand{\6}{\mbox{$[\hspace*{-.6mm}[$}} 
\newcommand{\9}{\mbox{$]\hspace*{-.6mm}]$}}
\newcommand{\sem}[2]{\6#1\9$_{#2}$}
\newcommand{\jl}{\langle}
\newcommand{\jr}{\rangle}
\newcommand{\ibr}{-\hspace*{-0.1cm}i}
\newcommand{\hbr}{-\hspace*{-0.1cm}h}


\def\bad{{\leavevmode\llap{*}}}
\def\marginal{{\leavevmode\llap{?}}}
\def\verymarginal{{\leavevmode\llap{??}}}
\def\infelic{{\leavevmode\llap{\#}}}

\begin{document}

SEMESTER \hfill Linguistik/Anglistik, University of Stuttgart

\begin{center}

{\bf \large COURSE NAME}


MEETING DAY, MEETING TIME
\\ 
MEETING LOCATION

\end{center}



{\bf Instructor:} INSTRUCTOR NAME
\\ Office: OFFICE NUMBER, K2  \\
Office hours: DAY, TIME, URL FOR ONLINE OFFICE HOURS \\
Email: EMAIL \\
Front office: Ralf Bothner, 4.057 K2, Tel.\ +49 711 685-83120

\medskip

{\bf Course topic:} 

DESCRIPTION OF COURSE CONTENT

\medskip

{\bf Learning objectives:} By the end of this class, you will have...

\begin{itemize}[topsep=-1ex,itemsep=-5pt]

\item LEARNING OBJECTIVE 1

\item \ldots

\end{itemize}

\medskip

{\bf Prerequisites:} PREREQUISITES

\medskip

{\bf Readings:} WHAT ARE THE REQUIRED READINGS? ARE THEY MADE AVAILABLE ON ILIAS?

\medskip

{\bf Assessment:} 

\begin{itemize}[topsep=-1ex,itemsep=-1pt]

\item ASSESSMENT CATEGORY 1 (USL/UVL?)

\item ASSESSMENT CATEGORY 2 (USL/UVL?)

\item \ldots

\end{itemize}

\bigskip

{\bf How we will interact in this (online version of the?) course:}

\begin{enumerate}[topsep=-3pt,itemsep=-1pt,leftmargin=15pt]

\item {\bf Email (via ILIAS):} I will send out email via ILIAS so please make sure to regularly read whichever email account you have connected to ILIAS.

\item {\bf Course design:} SYNCHRONOUS COURSE? FLIPPED COURSE? 

\item {\bf Reading questions:} To help you prepare for the in-class discussion of the assigned readings, I will make available reading questions, which we discuss in class. I will send these out via email a week before the relevant class meeting.

\item {\bf ILIAS fora:} There is a forum on ILIAS for each of the articles. If you have a question and you post it by 9pm on the day before we discuss the reading, we can address your question in class. I expect all of you to try to answer the questions that your fellow classmates post (before and after the relevant class meeting).

\item {\bf Study groups:} I highly recommend that you form study groups. If you cannot participate in one of the synchronous course meetings, please review the information posted to ILIAS (if any) and find out about the meeting content from your classmates, including those in your study group. 

\item {\bf Office hours:} See above.

\end{enumerate}

\newpage

{\bf Course schedule:}

\begin{longtable}{l p{15cm}}
{\bf Date} & {\bf Topic} \\ \hline

DATE & Syllabus  \\

DATE & \ldots \\ 

\multicolumn{2}{l}{\bf MODULE 1} \\ 

DATE & \ldots \\ 

DATE & \ldots \\ 

DATE & \ldots \\ 

& \\

\multicolumn{2}{l}{\bf MODULE 2} \\ 

DATE & \ldots \\

& \\

\hline\hline

DATE & EXAM DATE / DUE DATE OF PAPER \\ 

\hline\hline

\end{longtable}

{\bf Deadlines, extensions and Wiederholungspr\"ufungen}

\begin{itemize}[topsep=-1ex,itemsep=-1pt]

\item Deadlines are strict. 

\item {\bf USL/UVL:} If you cannot submit work for an USL/UVL by the deadline indicated in this syllabus, please get in touch with me {\bf before} the deadline, to discuss an alternative plan. 

\item {\bf Modulpr\"ufung:} For graded assessment components that you register on C@mpus (e.g., final exam, paper), your Pr\"ufungsordnung provides guidance on what to do in case you cannot meet the deadline or have missed the deadline. Typically, the following is the case (though you should check your Pr\"ufungsordnung!):

\begin{itemize}[leftmargin=2ex,topsep=-1ex,itemsep=-1pt]

\item You can withdraw from a Modulpr\"ufung up to 7 days before the due date. According to the Corona-VO, you can withdraw from in person (oral or written) exams up to 1 day before the due date. 

\item If you miss a Modulpr\"ufung that you are registered for, the grade is ``nicht ausreichend''.

\item If a) you missed the deadline to withdraw from a Modulpr\"ufung or b) you failed to withdraw from a Modulpr\"ufung and received a ``nicht ausreichend'', please get in touch with the chair of the Pr\"ufungsausschuss (Prof.\ Priewe) and the examiner to see if you can a) still withdraw or b) withdraw retroactively. 

\item To withdraw for reasons of illness, a doctor's note has to be submitted to the chair of the Pr\"ufungsausschuss that includes information on how long you are not able to participate in the exam. 

\item Taking care of a sick child or other close family member may also allow you to withdraw from a Modulpr\"ufung.

\end{itemize}

\item {\bf For this course}, there is a Zweittermin (if you cannot make the above deadline) and Wiederholungspr\"ufung (if you fail the module) on July 1, 2021. 

\end{itemize}

\bigskip

{\bf Academic misconduct:}
\\
I expect all the work you do in this course to be your own, unless
collaboration is explicitly requested for a particular task. Academic dishonesty will not be
allowed under any circumstances. Any case of cheating or plagiarism
will be handled according to academic policy (\url{https://www.student.uni-stuttgart.de/pruefungsorganisation/document/Plagiarism_Prevention_Guidelines.pdf}).

\medskip

{\bf Technical requirements:}
\\
You need to access ILIAS and WebEx, i.e., you need a computer that is connected to the internet. To fully participate in the WebEx meetings, you need a microphone and a camera. If your access to a computer that is connected to the internet is limited, please let me know and we will discuss accommodations.

\medskip
{\bf Data protection and privacy:}
\\
I will not record the online course meetings and you may also not record them. You are not required to turn on your microphone or camera in the online meetings. 

\medskip

{\bf Accommodations:}
\\
The University of Stuttgart strives to make all learning experiences accessible for students with disabilities or chronic illnesses. If you anticipate or experience academic barriers based on your disability or chronic illness, please let me know as soon as possible so that we can privately discuss options. For accommodations concerning exams please contact the chair of the Pr\"ufungsausschuss (\url{https://www.student.uni-stuttgart.de/beratung/pruefungsausschuss/}). Further information about accommodations and contact information for advisory services can be found here: \url{https://www.student.uni-stuttgart.de/en/counseling/disability/}.

\bigskip

{\bf References}

REFERENCES FOR THE READINGS

\bibliographystyle{/Users/tonhauser.1/Library/Latex/cslipubs-natbib}
\bibliography{/Users/tonhauser.1/Documents/bibliography}


\end{document}
