\documentclass[a4,11pt]{article}
\usepackage{tikz} %to draw models
\usetikzlibrary{tikzmark}
\usepackage{multicol} %for multi columns
\usepackage[utf8]{inputenc}
\usepackage{gb4e}
\usepackage{enumitem}
\usepackage{amsmath} %for fractions
\usepackage{fullpage}
\usepackage{graphicx}
\graphicspath{ { } }


\setlength{\parindent}{0in}

\definecolor{Pink}{RGB}{240,0,120}
\newcommand{\jt}[1]{\textbf{\textcolor{Pink}{#1}}}

% Semantic brackets
\newcommand{\6}{\mbox{$[\hspace*{-.6mm}[$}} 
\newcommand{\9}{\mbox{$]\hspace*{-.6mm}]$}}
\newcommand{\sem}[2]{\6#1\9$^{#2}$}

\title{Week 7: Quiz questions and model answers}
\author{Judith Tonhauser and Elena Vaik\v snorait\.{e} }
\date{\today}


\begin{document}

\maketitle

{\bf Introductory message:} This quiz covers the material in K 4.1-4.2 and CC sections 3.1 and 3.2. These sections serve as an introduction to propositional logic. It covers atomic and complex formulae and connectives that combine with formulae. The first two questions refer back to material covered earlier, namely the interpretation of quantifiers. Question 2 also allows you to practice writing an argument: make sure to answer all parts of the question and to refer back to the definition in formulating your argument.

\begin{enumerate}[leftmargin = 12pt]

\item {\bf Quantifier {\em every} with empty left argument/restriction 1:}  The word ``married'' denotes the set of married people and the word ``men'' denotes the set of men. If we form the complex expression ``married men'', we do so by intersecting the set of married people with the set of men, to get the set of married men as the meaning of ``married men''. By the same rule, the meaning of the expression ``married bachelor'' is determined by intersecting the set of married people with the set of men who are not and have never been married. Because this intersection is empty (one cannot be married and not married at the same time), the meaning of ``married bachelor'' is the empty set. Now consider the example ``Every married bachelor lives in New York''. What is your intuition about this sentence? Is it true? False? Or can you not say? (Every answer is correct here since we are asking about your intuition!)

\begin{itemize}[noitemsep]
\item true (1pt)
\item false (1pt)
\item can't say (1pt)
\end{itemize}

{\bf JT's answer:} According to my intuition this sentence is true but I've probably been too influenced by my logic education. A recent (as yet unpublished) experiment conducted with native speakers of American English found that many judge examples with empty left arguments/restrictions to be false, some say that they can't assess their truth, and only very few people judge such examples to be true.

\item {\bf Quantifier {\em every} with empty left argument/restriction 2:} Now let's consider what the semantics of {\em every} that was introduced in CC p.67 predicts. What does this semantics predict for the example ``Every married bachelor lives in New York'' and why? (max.\ 550 characters; 6 points)

{\bf Model answer:} According to CC p.67, ``Every A is a B'' is true if and only if every member of A is a member of B. Thus, ``Every married bachelor lives in New York'' is true if and only if every member of the set of married bachelors is a member of the set of people living in New York. Because the set of married bachelors is empty, it is trivially the case that every member of that set is a member of the set of people living in New York. Therefore, the semantics introduced in CC predicts that ``Every married bachelor lives in New York''  is true.

\item {\bf Propositional logic 1:} What are propositions without any logical connectives called?  (1 point)

\begin{itemize}[noitemsep]
\item atomic formulae
\item complex formulae
\end{itemize}

{\bf Model answer:} The correct answer is atomic formulae. Complex formulae are formed from atomic formulae with the help of connectives.

\item {\bf Propositional logic 2:} There are two types of logical connectives: binary and unary. Match each connective to labels binary or unary (correct match 1 point, incorrect match -1 point).

\begin{enumerate}[noitemsep]
\item $\land$
\item $\lor$
\item $\neg$
\item $\rightarrow$
\end{enumerate}

{\bf Model answer:} The correct answers are $\land$, $\lor$, $\rightarrow$ are binary, $\neg$ is a unary connective. Binary connectives apply two formulae together, while a unary connective applies to a single formula.

\item {\bf Syntax of propositional logic:} The syntactic rules for negation (p.101), conjunction (p.102) and disjunction (p.103) specify the rules by which one can create more complex expressions in propositional logic. Assume a propositional logic with the atomic formulae $u$, $v$ and $w$ (remember that any letter can be used for the atomic formulae). Which of the following expressions are syntactically well-formed according in propositional logic? Select all that are well-formed.

\begin{enumerate}[noitemsep]
\item $u$ \hspace*{2cm}% (1 checked / -1 unchecked)
\item $a$ \hspace*{2cm} %(- 1 checked / 1 unchecked)
\item $[u \wedge w]$ \hspace*{2cm} %(1 checked / -1 unchecked)
\item $[u \vee u]$ \hspace*{2cm} %(1 checked / -1 unchecked)
\item $u \wedge v \vee u$ \hspace*{2cm} %(- 1 checked / 1 unchecked)
\item $[[u \wedge v] \vee u]$ \hspace*{2cm} %(1 checked / -1 unchecked)
\item $\neg v$ \hspace*{2cm} %(1 checked / -1 unchecked)
\item $\neg [v \vee w]$ \hspace*{2cm} %(1 checked / -1 unchecked)
\item $\neg [v \vee \neg w]$ \hspace*{2cm} %(1 checked / -1 unchecked)
\item $\neg [v \neg \vee w]$ \hspace*{2cm} %(- 1 checked / 1 unchecked)
\item $\neg \neg \neg v$ \hspace*{2cm}% (1 checked / -1 unchecked)
\end{enumerate}

{\bf Model answer:} The expressions in (a), (c), (d), (f), (g), (h), (i) and (k) are well-formed.

(b) is not well-formed because $a$ is not a propositional letter. 

(e) is not well-formed because the syntactic rules for conjunction and disjunction specify that the two conjuncts and the two disjuncts are identified by brackets. This serves to disambiguate complex formulae of propositional logic. In (e), we don't know, for instance, whether the right conjunct is just $v$ or the complex expression $v \vee u$. The expression in (f) is a well-formed version of the expression in (e). In practice, we often leave out the outer brackets because they do not serve to further disambiguate the expression, i.e., instead of $[[u \wedge v] \vee u]$ we would write $[u \wedge v] \vee u$.

(j) is not well-formed because the negation applies not to a well-formed formula but rather to $\vee w$, which is not a formula, and therefore cannot serve as the argument for negation.


\item {\bf Semantics of propositional logic:} The semantic rules for propositional letters (p.99), for negation (p.102), for conjunction (p.103) and disjunction (p.104) specify the rules by which one can determine the truth value of atomic and complex formulae. Assume a propositional logic with the atomic formulae $u$, $v$ and $w$ and the two interpretation functions I1 = $\{\langle u, 1\rangle, \langle v, 1\rangle, \langle w, 0\rangle\}$ and I2 = $\{\langle u, 0\rangle, \langle v, 1\rangle, \langle w, 1\rangle\}$. Which of the following expressions are true? Select all that are true.


\begin{enumerate}[noitemsep]
\item \sem{$u$}{I1} \hspace*{2cm} 
%(2 checked / -2 unchecked)

\item \sem{$u$}{I2} \hspace*{2cm} %(-2 checked / 2 unchecked)


\item \sem{$u \wedge v$}{I1} \hspace*{2cm} %(2 checked / -2 unchecked)

\item \sem{$u \wedge v$}{I2} \hspace*{2cm} %(-2 checked / 2 unchecked)


\item \sem{$\neg w$}{I1} \hspace*{2cm} %(2 checked / -2 unchecked)

\item \sem{$\neg w$}{I2} \hspace*{2cm} %(-2 checked / 2 unchecked)


\item \sem{$[u \wedge v] \vee w$}{I1} \hspace*{2cm} %(2 checked / -2 unchecked)

\item \sem{$[u \wedge v] \vee w$}{I2} \hspace*{2cm} %(2 checked / -2 unchecked)

\end{enumerate}

{\bf Model answer:} The expression in (a) is true because the interpretation function I1 maps $u$ to 1, i.e., true. The expression in (b), on the other hand, is false because the interpretation function I2 maps $u$ to 0, i.e., false.

The expression in (c) is true because I1 maps both $u$ and $v$ to 1, and the complex expression $u \wedge v$ is true (1) according to the semantic rule for conjunction if and only if both of its arguments are true (which is the case here). The expression in (d) is false because I2 maps $u$ to false (0), and so it is not the case that both conjuncts are true.

The expression in (e) is true: I1 maps $w$ to 0, but the semantic rule for negation states that a complex negated expression is true if and only if the not-negated expression is false. Because $w$ is false under I1, its negation is true under I1. The expression in (f) is false: I2 maps $w$ to 1 (true), and so the negation of $w$ under I2 is false.

The expression in (g) is true: we already determined that the expression $u \wedge v$ is true under I2, and the semantic rule for disjunction states that the complex expression with disjunction is true if and only if at least one of the disjuncts is true. It doesn't matter that $w$ is false under I1 because the first disjunct is true. The expression in (h) is also true: even though the left disjunct is false (as we determined above) under I2, the right disjunct (i.e., $w$) is true under I2, which means that at least one disjunct is true, rendering the entire expression true.


\item {\bf Translating English into Propositional Logic:} Assume that the propositional letter K stands for the English sentence ``Kim sang'', the propositional letter S stands for the English sentence ``Sandy sang'' and the propositional letter A stands for ``Alex sang''. Which of the following propositional logic formulae are translations of the following complex English sentences? (correct match 2 point, incorrect match -2 point).

English sentences:

\begin{enumerate}[noitemsep]
\item Kim sang and Sandy sang. 
\item Kim sang or Alex sang.
\item Kim sang and Sandy didn't sing.
\item Kim didn't sing and Alex sang.
\item Neither Kim nor Alex sang
\end{enumerate}

Propositional logic formulae:

\begin{enumerate}[noitemsep]
\item $K \wedge S$
\item $K \vee A$
\item $K \wedge \neg S$
\item $\neg K \wedge A$
\item $\neg (K \vee A)$
\end{enumerate}

{\bf Model answer:} The English sentences in (a) - (e) are translated by the Propositional logic formulae in (a) - (e), respectively (i.e., in the order given).


\item {\bf Argument Template:} An argument that has the following form is called: (1 point)

\begin{enumerate}[noitemsep]
\item[] If P then Q.  \hfill (Premise 1)
\item[] Not Q.  \hfill  (Premise 2)
\item[] Therefore, P. \hfill  (Conclusion)
\end{enumerate}

\begin{enumerate}[noitemsep]
\item Modus Ponens
\item Modus Tollens
\item Denying the antecedent
\item Affirming the consequent
\end{enumerate}

{\bf Model answer:} The correct answer is \textit{Modus Tollens}.

\item {\bf Modus tollens}: Ex.3 on p. 96. For both of the following argument forms, say whether it is a valid argument or a fallacy.  (correct match 3 points; incorrect match -3 points)

{\bf Model answer:} The argument that has the form of {\it Modus Tollens} is valid, because it is not possible for the conclusion to be false if the premises are true, i.e. it's not possible that it rained last night. The argument that has the form of {\it Denying the antecedent} is a fallacy, because it is possible for the conclusion to be false even if the premises are true. Even if it did not rain last night, the lawn could be wet for some other reason, for example, a sprinkler went off last night.


\item {\bf De Morgan's Laws 1} One of the De Morgan's laws states that the two formulae below are equivalent. 

\begin{enumerate}
\item  $\neg [ P \land  Q ]$
\item  $ [ \neg P \lor \neg Q ]$
\end{enumerate}

State when two fomulae are equivalent and explain how the truth table below shows that  $\neg [ P \land  Q ]$
and $ [ \neg P \lor \neg Q ]$ are equivalent. (max.\ 350 characters; 3 points; 1 point for stating when formulae are equivalent; 1 point for stating explicitly which columns in the table need to be compared; 1 point for conclusion)

\begin{tabular}{c | c | c | c | c | c | c}
\hline \hline
P & Q & P $\land$ Q & $\neg$ (P $\land$ Q) & $\neg$ P & $\neg$ Q & ($\neg$ P $\lor$ $\neg$ Q) \\
\hline \hline
T & T & T  & F & F & F & F\\
T & F & F & T & F & T & T\\
F & T & F& T & T & F & T\\
F & F & F& T& T & T & T\\
\hline \hline


\end{tabular}

{ \bf Model answer:} Two formulae are equivalent if they are have the same truth values under the exact same circumstances. $\neg$ (P $\land$ Q) is True whenever ($\neg$ P $\lor$ $\neg$ Q) is True, namely when either P or Q or both P and Q are false. $\neg$ (P $\land$ Q) is False whenever  ($\neg$ P $\lor$ $\neg$ Q) is False, namely when P and Q are true. Since the two formulae are true/false under the same circumstances, they are equivalent.




\item {\bf De Morgan's Laws 2} Are following formulae are equivalent? Select the correct values for the column and whether the formulae are equivalent or not. (2 points)

 $[P \lor Q]; \neg [ \neg P \land \neg Q]$


\begin{tabular}{c | c | c | c }
\hline \hline
P & Q & P $\lor$ Q & $\neg [ \neg P \land \neg Q]$ \\
\hline \hline
T & T & T \\
T & F & T \\
F & T & T \\
F & F & F \\
\hline \hline
\end{tabular}

\begin{enumerate}
\item T T T F equivalent
\item T T T F not equivalent
\item T F T F equivalent
\item T F T F not equivalent
\item T F T F equivalent
\item F F T F not equivalent
\end{enumerate}


{ \bf Model answer:} The correct answer is {\it T T T F equivalent}. The formulae are equivalent because they have the same truth values under every assignment.

\item {\bf Entailment 1:} Ex. 7 p.96; Does P entail $[P \land Q]$? (4 points)

\begin{enumerate}[noitemsep]
\item Yes, because under each assignment that $[P \land Q]$ is True P is also True.
\item Yes, because under each assignment that P is True $[P \land Q]$ is also True.
\item No, because it is not the case that under each assignment that P is True $[P \land Q]$ is also True.
\item No, because it is not the case that under each assignment that $[P \land Q]$ is True P is also True.
\end{enumerate}

{ \bf Model answer:}  The correct answer is (c). The definition of entailment is {\it A entails B iff whenever A is true, B is also true}. P does not entail $[P \land Q]$, because $[P \land Q]$ is false when P is true and Q is false (as shown in the second row of the table below).

\begin{tabular}{c | c | c }
\hline \hline
P & Q &  $[P \land Q]$  \\
\hline
T & T & T \\
 T & F & F \\
F & T & F \\
F & F & F \\
\hline \hline
\end{tabular}

\item  {\bf Entailment 2:} Does $\neg$ P entail $[ \neg P \lor \neg Q ]$? (4 points)

\begin{enumerate}[noitemsep]
\item Yes, because under each assignment that $[ \neg P \lor \neg Q ]$ is True $\neg$ P is also True.
\item Yes, because under each assignment that $\neg$ P is True $[ \neg P \lor \neg Q ]$ is also True.
\item No, because it is not the case that under each assignment that $\neg$ P is True $[ \neg P \lor \neg Q ]$ is also True.
\item No, because it is not the case that under each assignment that $[ \neg P \lor \neg Q ]$ is True $\neg$ P is also True.
\end{enumerate}

{ \bf Model answer:}  The correct answer is (b). The definition of entailment is {\it A entails B iff whenever A is true, B is also true}. $\neg$ P entails $[ \neg P \lor \neg Q ]$, because whenever  $\neg$ P is True,  $[ \neg P \lor \neg Q ]$ is also true as shown in the first two rows of the table below.

\begin{tabular}{c | c | c }
\hline \hline
 $\neg$ P & $\neg$ Q & $[ \neg P \lor \neg Q ]$  \\
\hline
T & T & T \\
T & F & T \\
F & T & T \\
F & F & F \\
\hline \hline
\end{tabular}

\item {\bf Ambiguity 1:} One of the reasons for adopting a logic to analyze natural language meaning that is mentioned in both CC and K is that natural language is ambiguous but logic, including propositional logic, can be defined to not be ambiguous. Assume a propositional logic in which the propositional letter K translates the English sentence ``Kim sings'', the propositional letter S translates the English sentence ``Sandy laughs'' and the propositional letter A translates the English sentence ``Alex cries''. Explain why the English sentence in A is ambiguous but the propositional logic formulae in B is not (max.\ 350 characters; 4 points).

A. Kim sings and Sandy laughs or Alex cries.

B. $k \wedge [s \vee a]$

{\bf Model answer:} A is ambiguous because we do not know whether the right disjunct of ``and'' is ``Sandy laughs'' or whether it is ``Sandy laughs or Alex cries''. B is not ambiguous because the brackets make clear that right disjunct is the complex expression $s \vee a$, i.e., the translation of the English sentence ``Sandy laughs or Alex cries''.

\item  {\bf Ambiguity 2:} The sentence below contains two connectives: the negation (\textit{not} realized as \textit{didn't}) and the conjunction (\textit{and}). The sentence is ambiguous between two readings (a and b). Match each reading to the formula that captures it. (correct match 2 points, incorrect match -2 points)

\begin{exe}
\ex Mary didn't visit London and Paris.
\begin{xlist}
\ex Mary visited neither London nor Paris.
\ex It is not the case that Mary visited both London and Paris.
\end{xlist}
\end{exe}

Formulae, where P is \textit{Mary visited London} and Q is \textit{Mary visited Paris}. 
\begin{enumerate}[noitemsep]
\item $\neg [P \land Q ]$
\item $[\neg P \land \neg Q]$
\end{enumerate}

{ \bf Model answer:}  The correct answers are $[\neg P \land \neg Q]$ corresponds to reading (a) and $\neg [P \land Q ]$ corresponds to reading (b). In (b), the negation outscopes the conjuction $\neg [P \land Q ]$, therefore the sentence with this reading is true if Mary visited either one of the cities or neither of them. In (b), the conjuction outscopes the negation. The sentence with this reading is true only if Mary did not visit London and she did not visit Paris.


\end{enumerate}
\end{document}
