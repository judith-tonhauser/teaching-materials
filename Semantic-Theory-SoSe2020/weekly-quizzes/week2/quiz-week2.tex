\documentclass[a4,11pt]{article}
\usepackage{xcolor} %for color
\usepackage[utf8]{inputenc}
\usepackage{gb4e}
\usepackage{enumitem}
\usepackage{amsmath} %for fractions
\usepackage{fullpage}


\setlength{\parindent}{0in}

\definecolor{Pink}{RGB}{240,0,120}
\newcommand{\jt}[1]{\textbf{\textcolor{Pink}{#1}}}


\title{Week 2: Quiz questions and model answers}
\author{Judith Tonhauser and Elena Vaik\v snorait\.{e} }
\date{\today}

\begin{document}

\maketitle

{\bf Introductory message:} This quiz covers the material in CC chapter 1.1. It is designed to assess your understanding of the three types of meanings/implications/inferences introduced (entailments, conversational implicatures, presuppositions) and two of the diagnostics (also referred to as tests) that we use to distinguish these three types of implications from one another: the defeasibility diagnostic and the projection diagnostic. The quiz also assesses your understanding of the two types of arguments introduced (valid ones and sound ones).

\begin{enumerate}[leftmargin = 12pt]

   \item {\bf Terminology:} An entailment is a type of 
   \begin{enumerate}[noitemsep]
       \item implicature (0 pts)
        \item implication (1 pt)
           \item presupposition (0 pts)
   \end{enumerate}
   
   {\bf Model answer:} The correct answer is `implication'. The term `implication' is a cover term for a meaning that arises from an uttered sentence.  Entailments, conversational implicatures and presuppositions are the three types of implications (or: the three types of meanings/inferences) that are introduced in the chapter.
 
    \item {\bf Terminology:} If sentence A entails sentence B, then sentence A...B.
   \begin{enumerate}[noitemsep]
       \item implicates (0 pts)
        \item implies (1 pt)
           \item presupposes (0 pts)
   \end{enumerate}
   
   {\bf Model answer:} The correct answer is `implies'. The verb `implies' is a cover term: if sentence A implies B, then B might be entailed by A, B might be conversationally implicated by A or B might be presupposed by A. The answers `implicates' and `presupposes' are not correct: if sentence A entails B, then sentence A does not conversationally implicate B. And if sentence A entails sentence B, it is possible but not necessary that sentence A presupposes B.
   
    \item {\bf Terminology:} Which of the following expressions are synonyms for `implication' in this textbook? Select all that apply.
 
   \begin{enumerate}[noitemsep]
       \item meaning (1pt)
        \item inference (1pt)
           \item presupposition (0pts)
           \item implicature (0pts)
   \end{enumerate}
   
   {\bf Model answer:} The terms `meaning' and `inference' are synonyms of `implication' in this textbook. They are all cover terms for the three types of meanings discussed: entailments, conversational implicatures and presuppositions. Because presuppositions and implicatures are types of implications, these terms are not synonyms for `implication'.
      
 \item {\bf Defining entailment:} Choose the correct word: A sentence A entails a sentence B iff whenever A is $\frac{true (1pt)}{false (0pts)}$, B is $\frac{true (1pt)}{false (0pts)}$ too. 
 
  {\bf Model answer:} The correct answer is `true' and `true', per the definition in (3) in CC ch.\ 1.1.

\item {\bf Defeasibility diagnostic:} Assume that the inference B arises from the sentence A. To identify whether B is entailed by A or conversationally implicated by A, we apply the defeasibility diagnostic. How does this diagnostic work?

 \begin{enumerate}[noitemsep]

       \item We combine sentence A with the negation of B. If the resulting example (e.g., A and not B) is self-contradictory, B is entailed by A. If the resulting example is not self-contradictory, B is not entailed by A. (1pt)

       \item We combine the negation of the sentence A with B. If the resulting example (e.g., Not A and B) is self-contradictory, B is entailed by A. If the resulting example is not self-contradictory, B is not entailed by A. (0pts)
       
   \end{enumerate}

  {\bf Model answer:} The correct answer is the one where we combine the sentence A (which gives rise to the inference B) with the negation of the inference (e.g., A and not B). We do not change the original sentence A in the defeasibility diagnostic because we want to diagnose whether B is entailed from A; the negation of A is not relevant for this diagnostic. We combine A with the negation of B because if A entails B, then an example like ``A and not B'' is self-contradictory. That's why we have to negate B.
 
\item {\bf Ingredients of the defeasibility diagnostic (contrary/contradictory) 1:} To apply the defeasibility diagnostic, we need to construct the negation of an inference, i.e., the negation of a sentence that expresses that inference. For this purpose, the term `the negation of a sentence' means a sentence that stands in a contradictory relation to the sentence, not a sentence that stands in a contrary relation to the sentence. This exercise is about the difference between the contradictory and the contrary relation. For the three sentences below identify whether they stand in contrary or contradictory relation to \textit{Everybody danced}.

\begin{enumerate}[noitemsep]
    \item \textit{It is not the case that everybody danced}.
    \begin{enumerate}
        \item Contrary (0pts)
        \item Contradiction (1pt)
    \end{enumerate}
     \item \textit{Not everybody danced.}
         \begin{enumerate}
        \item Contrary (0pts)
        \item Contradiction (1pt)
    \end{enumerate}
     \item \textit{Nobody danced.}
         \begin{enumerate}
        \item Contrary (1pt)
        \item Contradiction (0pts)
    \end{enumerate}
\end{enumerate}

 {\bf Model answer:} The correct answer is that {\em It is not the case that everybody danced} and {\em Not everybody danced} stand in contradictory relation to {\em Everybody danced}: it is not possible for these two sentences to be true when {\em Everybody danced} is true and it is also not possible for these two sentences to be false when {\em Everybody danced} is false. The sentence {\em Nobody danced}, on the other hand, stands in a contrary relation to {\em Everybody danced}: while it is not possible for both sentences to be true at the same time, it is possible for both to be false at the same time. To see that they can both be false at the same time, imagine a party at which 3 of the 5 people danced. Then both {\em Nobody danced} and {\em Everybody danced} are both false.
 

\item {\bf Ingredients of the defeasibility diagnostic (contrary/contradictory) 2:} To apply the defeasibility diagnostic, we need to construct the negation of an inference, i.e., the negation of a sentence that expresses that inference. For this purpose, the term `the negation of a sentence' means a sentence that stands in a contradictory relation to the sentence, not a sentence that stands in a contrary relation to the sentence. This exercise is about the difference between the contradictory and the contrary relation. For the two sentences below determine whether they stand in contrary or contradictory relation to \textit{I always dance}.

\begin{enumerate}[noitemsep]
    \item \textit{It is not the case that I always dance.}
        \begin{enumerate}
        \item Contrary (0pts)
        \item Contradiction (1pt)
    \end{enumerate}
     \item \textit{I never dance.}
         \begin{enumerate}
        \item Contrary (1pt)
        \item Contradiction (0pts)
    \end{enumerate}
\end{enumerate}

 {\bf Model answer:} The correct answer is that {\em It is not the case that I always dance} stands in contradictory relation to {\em I always dance}: it is not possible both sentences to be true at the same time and it is also not possible for both sentences to be false at the same time. The sentence {\em I never dance}, on the other hand, stands in a contrary relation to {\em I always dance}: while it is not possible for both sentences to be true at the same time, it is possible for both to be false at the same time. To see that they can both be false at the same time, imagine that, of all the parties I go to, I dance at half of them. Then both {\em I always dance} and {\em I never dance} are false.
  
\item {\bf Ingredients of the defeasibility diagnostic (sentence construction):} Let's assume you want to diagnose whether sentence A entails sentence B with the defeasibility diagnostic. Which sentence would you construct, to collect judgments on whether it is self-contradictory?

\begin{exe}
\exi{}
\begin{xlist}
\exi{(A)} Every dog barked.
\exi{(B)} Every chihuahua barked.
\end{xlist}
\end{exe}
\begin{enumerate}
    \item Every dog barked. \#In fact, not every chihuahua barked. (1pt)
    \item Every dog barked. \#In fact, no chihuahua barked. (0pts)
\end{enumerate}

{\bf Model answer:} The correct answer is {\em Every dog barked. \#In fact, not every chihuahua barked.}: in this example, the second sentence stands in contradictory relation to sentence B. In the incorrect answer, {\em Every dog barked. \#In fact, no chihuahua barked}, the second sentence stands in contrary relation to sentence B. The defeasibility diagnostic requires the construction of a sentence that stands in contradictory relation to the sentence that conveys the implication to be tested: here, the implication we want to test is the one conveyed by the sentence {\em Every chihuahua barked}. 

\item {\bf Ingredients of the defeasibility diagnostic (judgment):} Let's assume that you constructed an example like {\em Every dog barked. In fact, not every chihuahua barked} to apply the defeasibility diagnostic. Which type of judgment do you now need to collect (from yourself or another speaker)?

\begin{enumerate}
    \item Is this example true? (0pts)
    \item Is this sentence grammatical? (0pts)
     \item Is this sentence self-contradictory? (1pt)
\end{enumerate}

{\bf Model answer:} The correct answer is `Is this sentence self-contradictory?'. The question of whether an example is true is also used in semantics, but it is used to assess truth relative to a model (this is the topic of chapter 1.2). The question of whether an example is grammatical belongs to the realm of syntax. 

\item {\bf Diagnosing implications using the defeasibility diagnostic:} Apply the defeasibility diagnostic: does sentence A entail sentence B?

\begin{exe}
\exi{}
\begin{xlist}
\exi{(A)} Every dog barked.
\exi{(B)} Every chihuahua barked.
\end{xlist}
\end{exe}
\begin{enumerate}
    \item Yes, sentence A entails sentence B. (1pt)
    \item No, sentence A does not entail sentence B. (0pts)
\end{enumerate}

{\bf Model answer:} The correct answer is `yes'. Sentence A entails sentence B because the example {\em Every dog barked. In fact, not every chihuahua barked} is self-contradictory.

     \item {\bf Defining conversational implicatures:} Is the following statement true or false? Conversational implicatures can be cancelled without producing a contradiction. 
   \begin{enumerate}[noitemsep]
        \item True (1pt)
        \item False (0pts)
    \end{enumerate}
    
      {\bf Model answer:} The correct answer is `true'. Conversational implicatures can be cancelled because they are not logically implied. This was illustrated in example (8) of CC ch.\ 1.1.

  
\item {\bf Diagnosing implications using the defeasibility diagnostic:} CC ch.1.1 discusses an example from the movie {\em When Harry met Sally}: in this context, the (utterance of the) sentence A implies B. The question is which type of implication B is relative to sentence A. To diagnose whether B is entailed by sentence A, we can use the defeasibility diagnostic. Which example should be judged for whether it is self-contradictory?

\begin{exe}
\exi{}
\begin{xlist}
\exi{(A)} She has a good personality. 
\exi{(B)} She is not attractive. 
\end{xlist}
\end{exe}
\begin{enumerate}
    \item She has a good personality, and she is attractive. (1pt)
    \item She has a good personality, and it is not the case that she has a good personality. (0pts)
    \item She has a good personality, and it is not the case that she is attractive. (0pts)
\end{enumerate}

{\bf Model answer:} The correct answer is {\em She has a good personality, and she is attractive}. In this example, sentence A is combined with the negation of B: {\em she is attractive} is a more natural way of saying {\em it is not the case that she is not attractive}. In the answer {\em She has a good personality, and it is not the case that she has a good personality},  sentence A is combined with its own negation! This does not serve to diagnose the type of implication that B is. In the answer {\em She has a good personality, and it is not the case that she is attractive}, sentence A is combined with the inference B: this also does not serve to diagnose the type of implication that B is: we already know that B is implied by A, i.e., sentence A is compatible with B, so we don't learn anything new by combining the two in this way.
 
 
\item {\bf Presuppositions are not defeasible:} As discussed in CC ch.1.1, the sentence {\em The theremin duo that Mozart wrote is very famous} presupposes that Mozart wrote a theremin duo. In other words, the meaning of the definite noun phrase {\em the theremin duo that Mozart wrote} is a presupposition. We can show that the presupposition is not defeasible using the defeasibility diagnostic. Which example should be judged for whether it is self-contradictory?

   \begin{enumerate}[noitemsep]
        \item The theremin duo that Mozart wrote is very famous, but it is not the case that Mozart wrote a theremin duo. (1pt)
        \item The theremin duo that Mozart wrote is very famous, but it is not the case that the theremin duo that Mozart wrote is very famous. (0pts)
        \item The theremin duo that Mozart wrote is very famous, but Mozart wrote a theremin duo. (0pts)
    \end{enumerate}
    
 {\bf Model answer:} The correct answer is {\em The theremin duo that Mozart wrote is very famous, but it is not the case that Mozart wrote a theremin duo}. In this example, sentence A ({\em The theremin duo that Mozart wrote is very famous}) is combined with the negation of the inference we want to diagnose (the inference is that Mozart wrote a theremin duo). In the answer {\em The theremin duo that Mozart wrote is very famous, but it is not the case that the theremin duo that Mozart wrote is very famous},  sentence A is combined with its own negation! This does not serve to diagnose the type of implication contributed by the definite noun phrase. In the answer {\em The theremin duo that Mozart wrote is very famous, but Mozart wrote a theremin duo}, sentence A is combined with the inference contributed by the definite noun phrase: this also does not serve to diagnose the type of inference: we already know that the inference arises form A, so we don't learn anything new by combining the two in this way.
 
 \item {\bf Entailments versus presuppositions (which diagnostic?):} As discussed in CC ch.1.1, both entailments and presuppositions are not defeasible. So the defeasibility diagnostic does not help to distinguish these two types of implications. Which diagnostic do we use to distinguish presuppositions from entailments?

   \begin{enumerate}[noitemsep]
        \item Projection diagnostic (1pt)
        \item Reinforcability diagnostic (0pts)
    \end{enumerate}
    
  {\bf Model answer:}  The correct answer is  `projection diagnostic'. The reinforcability diagnostic distinguishes conversational implicatures from non-defeasible inferences (entailments and presuppositions).
    
 \item {\bf The projection diagnostic:}  Assume that B is a non-defeasible inference of sentence A. So we know that B is not a conversational implicature of sentence A, but B may either be an ordinary entailment or a presupposition. To identify whether B is presupposed by A, we apply the projection diagnostic. How does this diagnostic work?
 
    \begin{enumerate}[noitemsep]

        \item We create versions of the sentence A: its negation, a polar question version and a conditional version. We then judge whether B is implied by these versions of A. If it is, then B is a presupposition. If it is not, then B is an ordinary entailment. (1pt)

\item We create versions of the inference B: its negation, a polar question version and a conditional version. We then judge whether these versions are implied by the sentence A. If they are, then B is a presupposition. If they are not, then B is an ordinary entailment. (0pts)

    \end{enumerate}

  {\bf Model answer:}  The correct answer is the one where we create versions of the sentence A, not the one where we create versions of the inference B. After all, we want to diagnose the status of the inference B relative to A: by creating entailment-canceling versions of A, we can assess whether the inference B is an ordinary entailment or a presupposition. 
 
 \item {\bf The projection diagnostic (versions of the original sentence):} Assume that you have already identified that the inference that Sam has sneezed before is a non-defeasible inference of the sentence {\em Sam sneezed again}. (You can test this using the defeasibility diagnostic!) Now you want to identify whether the inference that Sam has sneezed before is an ordinary entailment or a presupposition of {\em Sam sneezed again}. Which variants of the sentence {\em Sam sneezed again} do you consider? (Select all that apply.)
 
    \begin{enumerate}[noitemsep]
        \item Sam did not sneeze again. (1pt)
         \item Who sneezed? (0pts)
           \item Why did Sam sneeze again? (0pts)
           \item If Sam sneezed again, then he needs to see a doctor. (1pt)
           \item Did Sam sneeze again? (0pts)
           \item Does Sam have allergies? (0pts)
    \end{enumerate}
  
{\bf Model answer:}  The correct answers are {\em Sam did not sneeze again}, {\em If Sam sneezed again, then he needs to see a doctor} and {\em Did Sam sneeze again?}. These are versions of the sentence {\em Sam sneezed again}: its negation, its polar question version and a conditional version. The answers {\em Who sneezed?} and {\em  Why did Sam sneeze again?} are versions of the original sentence but they are not relevant for diagnosing presuppositions because they are content questions rather than the polar question version. The question {\em Does Sam have allergies?} is not a version of the original sentence, so it is also not relevant for diagnosing the status of the inference that Sam has sneezed before relative to the sentence {\em Sam sneezed again}.
  
  \item {\bf Applying the projection diagnostic:} Let's assume that your friend Colin judges that the inference that Sam has sneezed before arises not just from {\em Sam sneezed again}, but also from {\em Sam did not sneeze again}, {\em If Sam sneezed again, then he needs to see a doctor} and {\em Did Sam sneeze again?}. What do you conclude?
   
    \begin{enumerate}[noitemsep]
        \item The inference is a presupposition. (1pt)
 	\item The inference is an ordinary entailment. (0pt)
    \end{enumerate}
  
{\bf Model answer:}  The correct answer is that the inference is a presupposition. If Colin judges that the inference arises not just from the original sentence but also from the variants where entailments ``normally go to die'' (as CC write on p.27), then the inference is not an ordinary entailment but a presupposition.

       
 \item {\bf Which type of meaning 1:} Relative to sentence A, what type of implication is B? To answer this question, you should first apply the defeasibility diagnostic to identify whether B is a conversational implicature and, if B is a non-defeasible inference of A, the projection diagnostic to identify whether B is an ordinary entailment or a presupposition. See Figure 1.1 in CC ch.1.1.
\begin{exe}
\exi{}
\begin{xlist}
\exi{(A)} The book that John bought was on sale.
\exi{(B)} John bought a book.
\end{xlist}
\end{exe}

 
   \begin{enumerate}[noitemsep]
        \item Entailment (0pts)
         \item Implicature (0pts)
           \item Presupposition (1pt)
    \end{enumerate}

{\bf Model answer:} The correct answer is presupposition. First, we apply the defeasibility diagnostic: because {\em The book that John bought was on sale, but John did not buy a book} is self-contradictory, B is a non-defeasible inference of sentence A. Next, we apply the projection diagnostic. The relevant versions of sentence A are the following: {\em The book that John bought was not on sale}, {\em Was the book that John bought on sale?} and {\em If the book that John bought was on sale, he will be happy}. Because sentence A implies inference B, and the versions of sentence A also imply B, B is not an ordinary entailment but a presupposition.
 
 \item {\bf Which type of meaning 2:} Relative to sentence A, what type of implication is B? To answer this question, you should first apply the defeasibility diagnostic to identify whether B is a conversational implicature and, if B is a non-defeasible inference of A, the projection diagnostic to identify whether B is an ordinary entailment or a presupposition. See Figure 1.1 in CC ch.1.1.
\begin{exe}
\exi{}
\begin{xlist}
\exi{(A)}  The flying saucer came yesterday.
\exi{(B)} The flying saucer has come sometime in the past.
\end{xlist}
\end{exe}
   \begin{enumerate}[noitemsep]
        \item Entailment (1pt)
         \item Implicature (0pts)
           \item Presupposition (0pts)
    \end{enumerate}

 {\bf Model answer:} The correct answer is entailment. First, we apply the defeasibility diagnostic: because {\em The flying saucer came yesterday, but the flying saucer has not come sometime in the past} is self-contradictory, B is a non-defeasible inference of sentence A. Next, we apply the projection diagnostic. The relevant versions of sentence A are the following: {\em The flying saucer did not come yesterday}, {\em Did the flying saucer come yesterday?} and {\em If the flying saucer came yesterday, Mulder was right.} Because the versions of sentence A do not imply B, B is not a presupposition. 

  
 \item {\bf Which type of meaning 3:} Relative to sentence A, what type of implication is B? To answer this question, you should first apply the defeasibility diagnostic to identify whether B is a conversational implicature and, if B is a non-defeasible inference of A, the projection diagnostic to identify whether B is an ordinary entailment or a presupposition. See Figure 1.1 in CC ch.1.1.

\begin{exe}
\exi{}
\begin{xlist}
\exi{(A)}  The flying saucer came again.
\exi{(B)} The flying saucer has come sometime in the past.
\end{xlist}
\end{exe}
   \begin{enumerate}[noitemsep]
        \item Entailment (0pts)
         \item Implicature (0pts)
           \item Presupposition (1pt)
    \end{enumerate}
    
{\bf Model answer:} The correct answer is presupposition. First, we apply the defeasibility diagnostic: because {\em The flying saucer the flying saucer has come sometime in the past} is self-contradictory, B is a non-defeasible inference of sentence A. Next, we apply the projection diagnostic. The relevant versions of sentence A are the following: {\em The flying saucer did not come again}, {\em Did the flying saucer come again?} and {\em If the flying saucer came again, Mulder will be happy.}  Because sentence A implies inference B, and the versions of sentence A also imply B, B is not an ordinary entailment but a presupposition.

 
\item {\bf Validity and soundness of arguments 1:} Is the following argument valid / sound?
\begin{exe}
\ex
\begin{xlist}
\ex All telephone-booths are blue.
\ex All blue items are time-travel devices.
\ex Therefore, all telephone-booths are time-travel devices.
\end{xlist}
\end{exe}
 \begin{enumerate}[noitemsep]
    \item Valid? 
    \begin{enumerate}[noitemsep]
   \item Yes (1pt) 
  \item No (0pts)
  \end{enumerate}
    \item Sound?
        \begin{enumerate}[noitemsep]
       \item  Yes (0pts) 
       \item No (1pt)
           \end{enumerate}
\end{enumerate}

 {\bf Model answer:} The correct answer is valid and not sound. The argument is valid because its premises (sentences a and b) entail the conclusion (sentence c). We can prove that by applying the defeasibility diagnostic. {\em All telephone-booths are blue and all blue items are time-travel devices, but not all blue telephone-booths are time-travel devices.} is self-contradictory. This argument is not sound because the premises are not true: there are telephone booths that are not blue and blue items are not time-travel devices.

\item {\bf Validity and soundness of arguments 2:} Is the following argument valid / sound?

\begin{exe}
\ex
\begin{xlist}
\ex 2 + 2 = 4.
\ex Therefore, Paris is in Europe.
\end{xlist}
\end{exe}
 \begin{enumerate}[noitemsep]
    \item Valid? 
            \begin{enumerate}[noitemsep]
       \item Yes (0pts) 
       \item No (1pt)
         \end{enumerate}
    \item Sound? 
                \begin{enumerate}[noitemsep]
       \item Yes (0pts) 
       \item No (1pt)
                \end{enumerate}
\end{enumerate}

{\bf Model answer:} The correct answer is not valid and not sound. The argument is not valid because its premise (sentences a) does not entail the conclusion (sentence b). We can prove that by applying the defeasibility diagnostic: {\em 2+2 = 4 and Paris is not in Europe} is not a contradiction. While the premise of the argument is true (2 + 2 is, in fact, 4), the argument is not sound. It is not sound because an argument can only be sound if meets two criteria: its premises are true and the argument is valid.

\item {\bf Validity and soundness of arguments 3:} Is the following argument valid / sound?
\begin{exe}
\ex
\begin{xlist}
\ex Copenhagen is either in Denmark or in the Netherlands.
\ex Copenhagen is not in the Netherlands.
\ex Therefore, Copenhagen is in Denmark.
\end{xlist}
\end{exe}
 \begin{enumerate}[noitemsep]
    \item Valid? 
            \begin{enumerate}[noitemsep]
       \item Yes (1pt) 
       \item No (0pts)
         \end{enumerate}
    \item Sound? 
                \begin{enumerate}[noitemsep]
       \item Yes (1pt) 
       \item No (0pts)
                \end{enumerate}
\end{enumerate}

 {\bf Model answer:} The correct answer is valid and sound. The argument is valid because its premises (sentences a and b) entail the conclusion (sentence c). We can prove that by applying the defeasibility diagnostic. {\em Copenhagen is either in Denmark or in the Netherlands and Copenhagen is not in the Netherlands but Copenhagen is not Denmark.} is self-contradictory. The argument is also sound because all of its premises are true. It is true that Copenhagen is either in Denmark or in the Netherlands and it is true that Copenhagen is not in the Netherlands. 
 


\end{enumerate}



\end{document}
