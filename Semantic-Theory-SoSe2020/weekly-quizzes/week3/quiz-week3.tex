\documentclass[a4,11pt]{article}
\usepackage{tikz} %to draw models
\usetikzlibrary{tikzmark}
\usepackage{multicol} %for multi columns
\usepackage{xcolor} %for color
\usepackage[utf8]{inputenc}
\usepackage{gb4e}
\usepackage{enumitem}
\usepackage{amsmath} %for fractions
\usepackage{fullpage}


\setlength{\parindent}{0in}

\definecolor{Pink}{RGB}{240,0,120}
\newcommand{\jt}[1]{\textbf{\textcolor{Pink}{#1}}}


\title{Week 3: Quiz questions and model answers}
\author{Judith Tonhauser and Elena Vaik\v snorait\.{e} }
\date{\today}

\begin{document}

\maketitle


{\bf Introductory message:} This quiz covers the material in CC sections 1.2.1 and 1.2.2. These sections introduce the concepts of truth-conditional semantics (including truth values and truth conditions) and model-theoretic semantics (including possible worlds and models), respectively. The quiz is designed to asses your understanding of the key notions discussed in these sections. 

\begin{enumerate}[leftmargin = 12pt]

  \item {\bf Terminology:}  In truth-conditional semantics, the meaning of a sentence is:
         \begin{enumerate}[noitemsep]
       \item its truth value (0pts) 
        \item its truth conditions (1pt)
   \end{enumerate}
   
 {\bf Model answer:} The correct answer is `its truth conditions' as per page 32. Classically, there are only two truth values, namely `true' and `false', i.e., a sentence is either true or false. So, if the meaning of a sentence was it's truth value, then all the sentences that we can create in any language would only have two meanings (namely true or false)! Since we know that sentences have many more meanings than just two, truth-conditional semantics does not assume that the truth value is the meaning of a sentence. 

\item {\bf Determining truth values in the actual world:} Assume the actual world as the model. For the following sentences, select them only if they are true. 

      \begin{enumerate}[noitemsep]
        \item Berlin is the capital of Germany. (1pt checked / -1pt unchecked)
        \item Germany is a country in South America. (-1pt checked / 1pt checked)
        \item Vilnius is the capital of Lithuania. (1pt checked / -1pt unchecked)
        \item Lithuania is a country in Africa. (-1pt checked / 1pt checked)
         \end{enumerate}
         
 {\bf Model answer:} The correct answer is (a) and (c). The sentences (b) and (d) are false in the actual world.
 
  \item {\bf Determining truth values in a state of affairs:} Assume a state of affairs in which there is a man called Homer, a woman called Marge. They have three children: Bart (a boy), Lisa (a girl) and Maggie (a girl). For the following sentences, select them only if they are true. 

      \begin{enumerate}[noitemsep]
        \item Bart is Maggie's brother. (1pt checked / -1pt unchecked)
        \item Homer is Maggie's brother. (-1pt checked / 1pt checked)
        \item Lisa is Marge's daughter. (1pt checked / -1pt unchecked)
        \item Lisa and Maggie are brothers. (-1pt checked / 1pt unchecked)
         \end{enumerate}
         
 {\bf Model answer:} The correct answer is (a) and (c). The sentences (b) and (d) are false in the world we asked you to assume. 
 
  \item {\bf Determining truth values in possible world:} Now assume a world that is like the actual world except that Bonn is the unique capital of Germany, there is a man Homer and his three siblings are called Bart (a boy), Lisa (a girl) and Maggie (a girl). For the following sentences, select them only if they are true. 

      \begin{enumerate}[noitemsep]
        \item Berlin is the capital of Germany. (-1pt checked / 1pt unchecked)
        \item Homer is Maggie's brother. (1pt checked / -1pt unchecked)
        \item Homer and Bart are brothers. (1pt checked / -1pt unchecked)
        \item Homer is Lisa's father. (-1pt checked / 1pt unchecked)
         \end{enumerate}
         
 {\bf Model answer:} The correct answers are (b) and (c): in this possible world, Homer is Maggie's brother and Homer and Bart are brothers. The sentences in (a) and (d) are false in this possible world (though, of course, they are true in the actual world or the world of the Simpsons TV show).

  \item {\bf State of affairs versus possible worlds:} What's the difference between a state of affairs and a possible world?

      \begin{enumerate}[noitemsep]
        \item A state of affairs is a complete specification of how things are whereas a possible world may leave things open. (0pts)
	\item A possible world is a complete specification of how things are whereas a state of affairs may leave things open. (1pt)
         \end{enumerate}
         
 {\bf Model answer:} The correct answer is (b), as per the discussion p.34.
          
      \item { \bf Distinguishing between truth conditions and truth values:} This chapter discusses the difference between truth conditions and truth values. Do you know the truth value or the truth conditions for the sentence or both? Check which one you \textit{do} know.

\begin{exe}
\ex It rained in Vilnius on April 25 1240.
\end{exe}

       \begin{enumerate}[noitemsep]
       \item truth conditions (1pt)
        \item truth value (0pts)
        \item truth value and truth conditions (0pts)
        \item neither (0pts)
         \end{enumerate}

{\bf Model answer:} The correct answer is `truth conditions'. You do not know what the weather was like on that day in Vilnius. There are no written historical records so you cannot find out the truth value of this sentence. Nonetheless you understand the meaning of the sentence. That is because you know its truth conditions. You know what the world would have to be like for this sentence to be true.  


   \item {\bf Proposition:}  A proposition is the set of possible worlds in which a sentence is true.  What is the proposition denoted by the sentence \textit{Simon is an undergraduate student}?
       \begin{enumerate}[noitemsep]
       \item the set of all possible worlds in which Simon is an undergraduate student (1pt)
        \item the actual world (0 pt)
    \end{enumerate}     
    
 {\bf Model answer:} The correct answer is (a), as per the definition of a proposition on p.34. We do not know whether  \textit{Simon is an undergraduate student} is true in the actual world. But we can imagine possible worlds in which the sentence is true.
    
      \item {\bf Proposition 1:}  Assume that there are 10 worlds: w1, w2, w3, w4, w5, w6, w7, w8, w9, w10. In w1, w3 and w5, Simon is a dog. In w2, Simon is an undergraduate student at the University of Stuttgart. In w4, Simon is an undergraduate student at Stanford University. And in w8, Simon is an undergraduate student at Harvard University. In w5, Simon has already finished his studies and is working as an engineer. In w7, Simon is a graduate student. And in worlds w9 and w10, Simon is a cat.     
      Given these worlds, what is the proposition denoted by the sentence \textit{Simon is an undergraduate student}?
      
       \begin{enumerate}[noitemsep]
       \item the set of worlds w1, w2, w3, w4, w5, w6, w7, w8, w9 and w10 (0pts)
       \item the set of worlds w2, w4 and w8 (1pt)
       \item the set of worlds w2, w4, w7 and w8 (0pts)
       \item the set of worlds w1, w2 and w3 (0pts)       
    \end{enumerate}     

        {\bf Model answer:} The correct answer is (b). In these worlds, Simon is an undergraduate student albeit at different universities.

\item {\bf Terminology: Models:}  In model-theoretic semantics, sentences are interpreted relative to a model. A model consists of:
       \begin{enumerate}[noitemsep]
       \item a domain and an interpretation (1pt)
        \item a domain and a possible world (0pts)
        \item a domain and the actual world (0pts)
         \end{enumerate}
            
        {\bf Model answer:} The correct answer is a domain and an interpretation as per definition on p.35: `A domain and interpretation taken together are called a model.' A domain consists of all objects in the world (e.g. individuals) and various properties (e.g.  being an astronaut, being happy, etc). An interpretation function assigns the objects to names, properties and truth values in different possible worlds. For instance, the sentence \textit{Beyonce is an astronaut} is assigned true in some worlds by an interpretation function and false in other worlds. 
    
 
      \item {\bf Proposition in a model:}  Now assume that the proposition denoted by the sentence  \textit{Simon is an undergraduate student} is the set of worlds w1, w3 and w5. In which model is this the case? Select all that apply. 
            
       \begin{enumerate}[noitemsep]
       \item Model M1 has 5 worlds in its domain: w1, w2, w3, w4 and w5. Simon is an undergraduate student in all even-numbered worlds in this model. (-1pt checked / 1pt unchecked)
       \item Model M2 has 5 worlds in its domain: w1, w2, w3, w4 and w5. Simon is an undergraduate student in all odd-numbered worlds in this model. (1pt checked / -1pt unchecked)
       \item Model M3 has 10 worlds in its domain: w1, w2, w3, w4, w5, w6, w7, w8, w9, w10. Simon is an undergraduate student in all odd-numbered worlds in this model. (-1pt checked / 1pt unchecked) 
       \item Model M4 has 10 worlds in its domain: w1, w2, w3, w4, w5, w6, w7, w8, w9, w10. Simon is an undergraduate students in worlds w1, w3 an w5, and a dog in the other 7 worlds. (1pt checked / -1pt unchecked)
    \end{enumerate}     
  
 {\bf Model answer:} The correct answers are (b) and (d). In these models, Simon is an undergraduate student in worlds w1, w3 and w5, and in no other worlds. The answer in (a) is incorrect because Simon is an undergraduate student in worlds w2 and w4; this set is not identical to the set consisting of worlds w1, w3 and w5. The answer in (c) is incorrect because Simon is an undergraduate student in the set of worlds w1, w3, w5, w7 and w9. This set is not identical to the set consisting of worlds w1, w3 and w5 (but rather subsumes it). 
         
\item {\bf Entailment revisited} In chapter 1.1, an entailment was defined as follows: sentence B is an entailment of sentence A if and only if whenever A is true, B is true. Consider the following two sentences A and B. Reassure yourself, using the defeasibility diagnostic and the projection diagnostic from chapter 1.1 that A indeed entails B.  Now assume a model in which there are 10 worlds (w1,...,w10). Assume that Jill owns a cat in all 10 worlds and that this cat is white in the even-numbered worlds and black in the odd-numbered worlds. What are the propositions expressed by A and B?

\begin{exe}
\exi{A:} Jill owns a white cat.
\exi{B:} Jill owns a cat.
\end{exe}

      \begin{enumerate}[noitemsep]
       \item A: the set of worlds w1, w3, w5, w7, w9. \\ B: the set of worlds w1, w2, w3, w4, w5, w6, w7, w8, w9, w10. (0pts)
       \item A: the set of worlds w2, w4, w6, w8, w10. \\ B: the set of worlds w1, w2, w3, w4, w5, w6, w7, w8, w9, w10. (1pt)
       \item A: the set of worlds w1, w2, w3, w4, w5, w6, w7, w8, w9, w10. \\ B: the set of worlds w1, w3, w5, w7, w9.  (0pts)
       \item A: the set of worlds w1, w2, w3, w4, w5, w6, w7, w8, w9, w10. \\ B: the set of worlds w2, w4, w6, w8, w10. (0pts)
    \end{enumerate}    

 {\bf Model answer:} The correct answer is (b). Here, the proposition expressed by A are those 5 even-numbered worlds in which Jill owns a white cat and the proposition expressed by B are all 10 worlds (because Jill owns a cat in all 10 worlds).
 
\item {\bf Entailment as a subset relation:} In chapter 1.1, an entailment was defined as follows: sentence B is an entailment of sentence A if and only if whenever A is true, B is true. Now that we know that sentences denote propositions, chapter 1.2 introduces a different way of characterizing entailment: sentence A entails sentence B if and only if the proposition denoted by A is a subset of the proposition denoted by B. 

Subset definition: A set S1 is a subset of a set S2 if and only if all elements of S1 are included in S2 (or: if and only if all members of S1 are also members of S2).

Consider again the two sentences A and B and a model with 8 worlds (w1,...,w8). If the proposition expressed by A is the set of worlds w1 and w2, i.e., A is true in worlds w1 and w2, and Jill does not own a cat in worlds w7 and w8, what is the proposition expressed by B?

\begin{exe}
\exi{A:} Jill owns a white cat.
\exi{B:} Jill owns a cat.
\end{exe}

      \begin{enumerate}[noitemsep]
       \item B: the set of worlds w1, w2, w3, w4, w5, w6 (1pt)
       \item B: the set of worlds w3, w4, w5, w6 (0pts)
       \item B: the set of worlds w1, w2, w7, w8 (0pts)
    \end{enumerate}    

 {\bf Model answer:} The correct answer is (a). This is the set of worlds in which Jill owns a cat: that cat is white in worlds w1 and w2, and has some other color in worlds w3-w6. The answer in (b) is incorrect because Jill also has a cat in worlds w1 and w2. The answer in (c) is incorrect because Jill does not have a cat in worlds w7 and w8. The proposition expressed by A is the set of worlds w1 and w2; this set is a subset of the proposition expressed by B, which is the set of worlds w1, w2, w3, w4, w5, w6.
     
      
      \item {\bf Models 1:} A proposition expressed by a sentence is the set of possible worlds in which the sentence is true. What is the proposition of expressed by the sentence \textit{Dan saw a giraffe} given the Model that contains four possible worlds (w1, w2, w3 and w4)? The arrows indicate whether a sentence is true or false in a particular world. Select all relevant worlds.
      

\begin{multicols}{2}

  \begin{tabular}{l  l l}
  \textbf{w1} && \\
    Dan saw a giraffe\tikzmark{a1}  &  & True\tikzmark{True1} \\
    Beyonce is an astronaut\tikzmark{b1}  & &   \\
   Mary likes Kyle\tikzmark{c1}  &  &  \\
   Dan saw a possum\tikzmark{d1}  &  &  \tikzmark{False1}False  \\
  \end{tabular}
  \begin{tikzpicture}[overlay, remember picture, yshift=.25\baselineskip, shorten >=.5pt, shorten <=.5pt]
    \draw [->] ({pic cs:a1}) [bend left] to ({pic cs:False1});
    \draw [->] ([yshift=.75pt]{pic cs:b1}) -- ({pic cs:False1});
      \draw [->] ({pic cs:c1}) [bend left] to ({pic cs:True1});
      \draw [->] ({pic cs:d1}) [bend left] to ({pic cs:True1});
  \end{tikzpicture}   

  
    \begin{tabular}{l  l l}
      \textbf{w2} && \\
    Dan saw a giraffe\tikzmark{a2}  &  & True\tikzmark{True2} \\
    Beyonce is an astronaut\tikzmark{b2}  & &   \\
   Mary likes Kyle\tikzmark{c2}  &  &  \\
   Dan saw a possum\tikzmark{d2}  &  &  \tikzmark{False2}False  \\
  \end{tabular}
  \begin{tikzpicture}[overlay, remember picture, yshift=.25\baselineskip, shorten >=.5pt, shorten <=.5pt]
    \draw [->] ({pic cs:a2}) [bend left] to ({pic cs:True2});
    \draw [->] ([yshift=.75pt]{pic cs:b2}) -- ({pic cs:True2});
      \draw [->] ({pic cs:c2}) [bend left] to ({pic cs:True2});
      \draw [->] ({pic cs:d2}) [bend left] to ({pic cs:True2});
  \end{tikzpicture}   
  
    \begin{tabular}{l  l l}
          \textbf{w3} && \\
    Dan saw a giraffe\tikzmark{a3}  &  & True\tikzmark{True3} \\
    Beyonce is an astronaut\tikzmark{b3}  & &   \\
   Mary likes Kyle\tikzmark{c3}  &  &  \\
   Dan saw a possum\tikzmark{d3}  &  &  \tikzmark{False3}False  \\
  \end{tabular}
  \begin{tikzpicture}[overlay, remember picture, yshift=.25\baselineskip, shorten >=.5pt, shorten <=.5pt]
    \draw [->] ({pic cs:a3}) [bend left] to ({pic cs:False3});
    \draw [->] ([yshift=.75pt]{pic cs:b3}) -- ({pic cs:False3});
      \draw [->] ({pic cs:c3}) [bend left] to ({pic cs:False3});
      \draw [->] ({pic cs:d3}) [bend left] to ({pic cs:False3});
  \end{tikzpicture}   
 
     \begin{tabular}{l  l l}
      \textbf{w4} && \\
    Dan saw a giraffe\tikzmark{a4}  &  & True\tikzmark{True4} \\
    Beyonce is an astronaut\tikzmark{b4}  & &   \\
   Mary likes Kyle\tikzmark{c4}  &  &  \\
   Dan saw a possum\tikzmark{d4}  &  &  \tikzmark{False4}False  \\
  \end{tabular}
  \begin{tikzpicture}[overlay, remember picture, yshift=.25\baselineskip, shorten >=.5pt, shorten <=.5pt]
    \draw [->] ({pic cs:a4}) [bend left] to ({pic cs:True4});
    \draw [->] ([yshift=.75pt]{pic cs:b4}) -- ({pic cs:False4});
      \draw [->] ({pic cs:c4}) [bend left] to ({pic cs:False4});
      \draw [->] ({pic cs:d4}) [bend left] to ({pic cs:False4});
  \end{tikzpicture}   
  
\end{multicols}
   
       \begin{enumerate}[noitemsep]
         \item w1 (-1pt checked / 1pt unchecked)
        \item w2 (1pt checked / -1pt unchecked)
       \item w3 (-1pt checked / 1pt unchecked)
       \item  w4 (1pt checked / -1pt unchecked)
        \end{enumerate}
   
 {\bf Model answer:} The correct answer is `w2, w4'. The proposition expressed by a sentence is the set of possible worlds in which the sentence is true. The sentence \textit{Dan saw a giraffe} is true in worlds 2 and 4. The sentence \textit{Dan saw a giraffe} is false in worlds 1 and 3. Therefore, the proposition expressed by  \textit{Dan saw a giraffe} is w2 and w4.
              
       \end{enumerate}

        


\end{document}
