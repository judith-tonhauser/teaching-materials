\documentclass[a4,11pt]{article}
\usepackage{tikz} %to draw models
\usetikzlibrary{tikzmark}
\usepackage{multicol} %for multi columns
\usepackage[utf8]{inputenc}
\usepackage{gb4e}
\usepackage{enumitem}
\usepackage{amsmath} %for fractions
\usepackage{fullpage}
\usepackage{graphicx}
\graphicspath{ { } }
\usepackage{tipa}


\def\bad{{\leavevmode\llap{*}}}
\def\marginal{{\leavevmode\llap{?}}}
\def\verymarginal{{\leavevmode\llap{??}}}
\def\infelic{{\leavevmode\llap{\#}}}

\setlength{\parindent}{0in}

\definecolor{Pink}{RGB}{240,0,120}
\newcommand{\jt}[1]{\textbf{\textcolor{Pink}{#1}}}

% Semantic brackets
\newcommand{\6}{\mbox{$[\hspace*{-.6mm}[$}} 
\newcommand{\9}{\mbox{$]\hspace*{-.6mm}]$}}
\newcommand{\sem}[2]{\6#1\9$^{#2}$}

\title{Week 10: Quiz questions and model answers}
\author{Judith Tonhauser and Elena Vaik\v snorait\.{e} }
\date{\today}


\begin{document}

\maketitle

{\bf Introductory message:} This quiz covers the material in K 20, except 20.4.2. Together with next week's chapter, this last part of the course is concerned with English tense and aspect.

\begin{enumerate}[leftmargin = 12pt]

\item { \bf `What happened?' diagnostic} The `what happened?' diagnostic distinguishes events and states. Apply the diagnostic to identify which of the following sentences denote events and which denote states. Select all of the events.

\begin{enumerate}[noitemsep]
\item Kim built a house. (1pt selected /  -1pt unselected)
\item Sandy sneezed. (1pt selected /  -1pt unselected)
\item Alex walked. (1pt selected /  -1pt unselected)
\item Taylor was frustrated. (-1pt selected /  1pt unselected)
\end{enumerate}

{ \bf Model answer:} According to the `what happened?' diagnostic, the sentences in (a)-(c) denote events because these sentences can, according to our judgment, be used in response to the question {\em What happened?}. The sentence in (d) denotes a state because it is odd, according to our judgment, in response to that question.

\item {\bf Progressive diagnostic} The progressive diagnostic distinguishes events and states. Apply the diagnostic to identify which of the following sentences denote events and which denote states. Select all of the events.

\begin{enumerate}[noitemsep]
\item Kim built a house. (1pt selected /  -1pt unselected)
\item Sandy sneezed. (1pt selected /  -1pt unselected)
\item Alex walked. (1pt selected /  -1pt unselected)
\item Taylor was frustrated. (-1pt selected /  1pt unselected)
\end{enumerate}

{\bf Model answer:} According to the progressive diagnostic, the sentences in (a)-(c) denote events because the progressive variants of these sentences (i.e., {\em Kim was building a house, Sandy was sneezing, Alex was walking}) are acceptable, according to our judgment. The sentence in (d) denotes a state because its progressive version (i.e., {\em Taylor was being frustrated}) sounds odd, according to our judgment.

\item {\bf `for'-adverb and `in'-adverb diagnostics} The `for'- and `in'-adverb diagnostics distinguish telic events from atelic situations. Apply the diagnostics to identify which of the following sentences denote telic events and which denote atelic situations. Select all of the telic events.

\begin{enumerate}[noitemsep]
\item Kim baked a cake. (1pt selected /  -1pt unselected)
\item Jordan reached the top of the mountain. (1pt selected /  -1pt unselected)
\item Cameron discovered the bug in the code. (1pt selected /  -1pt unselected)
\item Sandy sneezed. (-1pt selected /  1pt unselected)
\item Alex sang. (-1pt selected /  1pt unselected)
\item Taylor was frustrated. (-1pt selected /  1pt unselected)
\end{enumerate}

{\bf Model answer:} According to the `for'-adverb diagnostic, the sentences in (a)-(c) denote telic events: variants of these sentences with a {\em for-}adverbial, like {\em for 5 minutes}, sound odd, according to our judgment.  The sentences in (d)-(f) denote atelic situations because variants of these sentences with a {\em for}-adverbial like {\em for 5 minutes} sound good, according to our judgment. 

According to the `in'-adverb diagnostic, the sentences in (a)-(c) denote telic events: variants of these sentences with an {\em in-}adverbial, like {\em in 5 hours}, sound fine, according to our judgment.  The sentences in (d)-(f) denote atelic situations because variants of these sentences with an {\em in}-adverbial like {\em in 5 hours} sound odd, according to our judgment. 

\item { \bf Situation types} Match the following 5 sentences to the 5 situation types, by applying diagnostics for distinguishing events from states, for identifying telic events and for distinguishing durative from punctual situation types. (each correct match 2pts)

Situation types

\begin{enumerate}[noitemsep]
\item state
\item activity
\item accomplishment
\item achievement
\item semelfactive
\end{enumerate}

Sentences

\begin{enumerate}[noitemsep]
\item Taylor loves apple pie.
\item Jordan danced wildly.
\item Kim demolished the car.
\item Cameron stepped into the bar.
\item Alex winked at Sam.
\end{enumerate}

{ \bf Model answer:} The sentence in (a) denotes a state. As shown in (1), (a) is unacceptable in response to a {\em What happened?} question (which means that it denotes a state), in the progressive (which supports the hypothesis that it denotes a state) and with an {\em in}-adverb (which is compatible with the hypothesis that it denotes a state). States often can co-occur with a {\em for-}adverbial but a present tense situation of somebody loving apple pie is typically interpreted habitually and therefore not temporally restricted, so the {\em for-}adverbial is odd here. In the past tense, the combination is acceptable, supporting the assumption that (a) denotes a state: {\em Taylor loved apple pie for 10 years, but now she doesn't anymore}.

The sentence in (b) denotes an activity. As shown in (2), (b) is acceptable in response to a {\em What happened?} question (which means that it denotes an event) and in the progressive (which supports the hypothesis that it denotes a durative event). It is acceptable with a {\em for-}adverb and unacceptable with an {\em in-}adverb, which supports the hypothesis that (b) denotes an atelic durative event, i.e., an activity. 

The sentence in (c) denotes an accomplishment. As shown in (3), (c) is acceptable in response to a {\em What happened?} question (which means that it denotes an event) and in the progressive (which supports the hypothesis that it denotes a durative event). It is acceptable with a {\em in-}adverb, which supports the hypothesis that it denotes a telic event. It is unacceptable with a {\em for-}adverb, which again supports the hypothesis that it is a telic durative event.

The sentence in (d) denotes an achievement. As shown in (4), (d) is acceptable in response to a {\em What happened?} question (which means that it denotes an event). It is unacceptable in the progressive, which either means that it is not an event after all or that it is a punctual event, i.e., an achievement. The fact that (d) is unacceptable with the {\em in-}adverb is compatible with either hypothesis: either (d) is a state or a punctual event. Given that stepping into a location, like a bar, is not a homogenous situation but involves a change, it is more plausible that (d) is a punctual event. The fact that (d) is acceptable with a {\em for-}adverb might at first be puzzling but this combination does not describe the time during which Cameron's stepping took place but for how long Cameron stayed in the bar, i.e., it measures the duration of the result state. If you took (d) to be acceptable in the progressive, then you would be justified in arguing that (d) denotes an accomplishment.

The sentence in (e) denotes a semelfactive. As shown in (5), (e) is acceptable in response to a {\em What happened?} question (which means that it denotes an event). It is acceptable in the progressive, but it receives an iterative interpretation according to which Alex winked multiple times. This suggests that it is a semelfactive rather than a durative event. The fact that (e) is unacceptable with the {\em in-}adverb is compatible with that hypothesis and speaks against the assumption that it is an accomplishment. And with the {\em for-}adverb, we again get an iterative interpretation, again supporting the assumption that (e) denotes a semelfactive, i.e., a punctual atelic event.

\begin{exe}
\ex
\begin{xlist}
\ex A: What happened?  B: \# Taylor loves apple pie.
\ex \infelic Taylor is loving apple pie.
\ex Taylor loves apple pie \#for / \#in one hour
\end{xlist}
\ex 
\begin{xlist}
\ex A: What happened?  B: Jordan danced wildly.
\ex Jordan was dancing wildly.
\ex Jordan was dancing wildly for / \#in an hour.
\end{xlist}
\ex
\begin{xlist}
\ex A: What happened?  B: Kim demolished the car.
\ex Kim was demolishing the car.
\ex Kim demolished the car \# for / in an hour.
\end{xlist}
\ex
\begin{xlist}
\ex A: What happened?  B: Cameron stepped into the bar.
\ex \infelic Cameron was stepping into the bar.
\ex Cameron stepped into the bar for / \#in an hour. 
\end{xlist}
\ex
\begin{xlist}
\ex A: What happened?  B: Alex winked at Sam.
\ex Alex was winking at Sam.
\ex Alex winked at Sam for / \#in 5 minutes.
\end{xlist}
\end{exe}

\item {\bf Diagnosing situation type} Apply the `what happened?', the progressive and the `in'-/`for'-adverbial diagnostics to diagnose the situation type of the sentence {\em Taylor discovered a cure}.  Make sure to provide relevant examples, to explicitly state what the examples show and to explicitly state what you are arguing for. (max.\ 650 characters; 8 points)

{\bf Model answer:} The sentence {\em Taylor discovered a cure} describes an achievement. The situation is an event because the sentence is acceptable in response to a {\em What happened?} question, as shown in (6a). The fact that the progressive variant of the sentence is unacceptable, as shown in (6b), means that it is either a state or a non-durative event. The fact that the sentence is acceptable with an {\em in-}adverb and unacceptable with a {\em for-}adverb, as shown in (6c), supports the assumption that the situation type is telic.

\begin{exe}
\ex
\begin{xlist}
\ex A: What happened?  B: Taylor discovered a cure.
\ex \infelic Taylor was discovering a cure.
\ex Taylor discovered a cure \#for / in one year.
\end{xlist}
\end{exe}

{\bf Addendum:} If you judge (6b) to be acceptable, then you would want to argue that the sentence describes an accomplishment rather than an achievement. That is fine! What is important is a) that you state what you are arguing for and b) that you provide empirical evidence in support of that claim.

\item {\bf Topic time and tense} As discussed in the textbook, the topic time of an utterance is the time that the utterance is about. The tense of the uttered sentence indicates the topic time: if the topic time is in the past, the sentence must have past tense; if the topic time is present, the sentence must have present tense; and if the topic time is in the future, the sentence must have future tense.  What are the topic times and tenses of the following English sentences? (2pts for each match)

Topic time or tense:

\begin{enumerate}[noitemsep]
\item past
\item present
\item future
\end{enumerate}

Sentences:

\begin{enumerate}[noitemsep]
\item Taylor loves apple pie.
\item Jordan danced wildly.
\item Kim is parking the car.
\item Alex will wink at Sam.
\item Cameron is taking the train later today.
\end{enumerate}

{\bf Model answer}: The tense of the sentences in (a), (c) and (e) is present, the tense of the sentence in (b) is past and the tense of the sentence in (d) is future. If you said that the tense of (e) is future, be careful not to confuse tense and aspect: the finite verb of the example in (e) is {\em is}, which is present tense (future would be {\em will be}). The fact that the situation in (e) is in the future is not due to tense but due to the progressive aspect, which is compatible with future situations. In general, tense is {\bf always} marked on the finite verb.

\item {\bf Grammatical aspect} What are the grammatical aspects of the following English examples? (2pts for each match)

Grammatical aspects

\begin{enumerate}[noitemsep]
\item perfective
\item habitual
\item progressive continuous
\item non-progressive continuous
\end{enumerate}

Sentences

\begin{enumerate}[noitemsep]
\item Jordan danced wildly.
\item Alex runs every morning.
\item Kim was parking the car.
\item Taylor has measles.
\end{enumerate}

{\bf Model answer}: The grammatical aspect of (a) is perfective: in English, this grammatical aspect is realized by the simple form of the verb. The grammatical aspect of (b) is habitual: in English, this grammatical aspect, too, can be realized by the simple form of the verb. In the past, the simple form is often interpreted as perfective and, in the present, as habitual. But that is not necessarily the case. For instance, the simple past form can also receive a habitual interpretation: in {\em Last year, Alex ran in the park}, the grammatical aspect is not perfective but habitual. So, the simple form of the English verb is compatible with perfective and habitual interpretations, and we have to go by meaning to distinguish the two.

The grammatical aspect of (c) is progressive: in English, this grammatical aspect is formed by a finite form of the verb {\em to be} (here: {\em was}) and the present participle (here: {\em parking}). The grammatical aspect of (d) is continuous: Taylor having measles is an ongoing state. This is not progressive grammatical aspect because that is formed, as just noted, with the finite form of {\em to be} and a present participle. Instead, here the stative situation type of having measles is ongoing and, as mentioned on p.391, the term `continous' is used for ongoing states.

\end{enumerate}
\end{document}
