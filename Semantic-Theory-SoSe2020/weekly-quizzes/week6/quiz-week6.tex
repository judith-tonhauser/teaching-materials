\documentclass[a4,11pt]{article}
\usepackage{tikz} %to draw models
\usetikzlibrary{tikzmark}
\usepackage{multicol} %for multi columns
\usepackage{xcolor} %for color
\usepackage[utf8]{inputenc}
\usepackage{gb4e}
\usepackage{enumitem}
\usepackage{amsmath} %for fractions
\usepackage{fullpage}
\usepackage{graphicx}
\graphicspath{ { } }


\setlength{\parindent}{0in}

\definecolor{Pink}{RGB}{240,0,120}
\newcommand{\jt}[1]{\textbf{\textcolor{Pink}{#1}}}


\title{Week 5: Quiz questions and model answers}
\author{Judith Tonhauser and Elena Vaik\v snorait\.{e} }
\date{\today}

\begin{document}

\maketitle

{\bf Introductory message:} This quiz covers the material in CC section 2.5. This section discusses relations and functions.

\begin{enumerate}[leftmargin = 12pt]

\item {\bf Ordered pairs versus sets:} Imagine that Kim admires Sandy. To represent this state of affairs, which is more appropriate: a representation using an ordered pair, as in $\langle$Kim, Sandy$\rangle$, or a representation using a set, as in $\{$Kim, Sandy$\}$?

 \begin{enumerate}[noitemsep]
        \item The English verb {\em admire} is transitive, which means that it expresses a relation between pairs of individuals. The ordered pair is more appropriate: it captures that Kim and Sandy stand in a specific relation to one another, whereas the set merely captures that Kim and Sandy both have some property. (1pt)
        
        \item The English verb {\em admire} is transitive, which means that it expresses a relation between pairs of individuals. The set is more appropriate: it captures that Kim and Sandy stand in a relation to one another, whereas the ordered pair merely captures that Kim and Sandy both have some property. (0pts)
\end{enumerate}

{\bf Model answer:} Answer (a) is correct. The meaning of {\em admire} is better captured as a set of pairs of individuals that stand in the `admire' relation to one another than as a set of individuals (or than a set of sets of individuals).

\item {\bf Exercise 11 on p.82:} Select the true statements.

{\bf Model answer:}  The following statements are true: (a), (b), (d), (f), (g) and (h). Statements (a), (b), (f), (g) and (h) are true because of the following two properties of sets: the elements of a set are not ordered and the elements of a set are unique (i.e., \{a\} = \{a, a\}). Statement (d) is true because the two objects to the left and the right of the equal sign are identical, namely the pair $\langle 3,3 \rangle$. 

Statement (c) is false because the pair $\langle 3,4\rangle$ has 3 as its first member and 4 as its second. That is not the case for the pair $\langle 4, 3 \rangle$.

Statement (e) is false because $\{ \langle 3,3 \rangle\}$ is a set with one element, namely the pair $\langle 3,3\rangle$. The expression to the right of the equal sign is not a set, but just the pair $\langle 3,3\rangle$.

\item {\bf Cartesian product:} Assume the sets A = $\{$Stuttgart, Paris, Asuncion, Washington$\}$  and  B = $\{$Germany, France, Paraguay, USA $\}$. What is the Cartesian product of A and B (A $\times$ B)? Select all that apply.

\begin{enumerate}[noitemsep]
        \item the set of all pairs $\langle$a,b$\rangle$ such that $a$ is an element of $A$ and $b$ is an element of $B$ (2 checked / -2 unchecked)

	\item the set \begin{minipage}[t]{7cm}  $\{\langle Stuttgart, Germany \rangle, \langle Stuttgart, France \rangle, \langle Stuttgart, Paraguay \rangle, \langle Stuttgart, USA \rangle$, $\langle Paris, Germany \rangle, \langle Paris, France \rangle, \langle Paris, Paraguay \rangle, \langle Paris, USA \rangle, \langle Asuncion, Germany \rangle$, $\langle Asuncion, France \rangle , \langle Asuncion, Paraguay \rangle, \langle Asuncion, USA \rangle, \langle Washington, Germany \rangle$, $\langle Washington, France \rangle, \langle Washington, Paraguay \rangle, \langle Washington, USA \rangle\}$ \end{minipage} \\ (2 checked / -2 unchecked)

        \item the set $\{\langle Stuttgart, Germany \rangle, \langle Paris, France \rangle, \langle Asuncion, Paraguay \rangle, \langle Washington, USA \rangle\}$ (-2 checked / 2 unchecked)
\end{enumerate}

\item {\bf Domain, codomain and range:} Assume the sets A = $\{ Stuttgart, Paris, Asuncion, Washington \}$  and  B = $\{ Germany, France, Paraguay, USA \}$ \\ and the relation R = $\{\langle Stuttgart, Germany \rangle, \langle Paris, France \rangle, \langle Asuncion, Paraguay \rangle\}$ in the Cartesian product of A and B. Which of the following go together? (1pt for each match)

Expression:

\begin{enumerate}[noitemsep]
        \item domain of R
        \item codomain of R
        \item range of R
\end{enumerate}

Sets:

\begin{enumerate}[noitemsep]
        \item $\{ Stuttgart, Paris, Asuncion, Washington \}$
        \item B
        \item $\{ Germany, France, Paraguay \}$
\end{enumerate}   

{\bf Model answer:} The domain of R is the set A = $\{ Stuttgart, Paris, Asuncion, Washington \}$. The codomain of R is the set B. The range of R are those elements of B that are the second member of a pair in R, so the set $\{ Germany, France, Paraguay \}$


\item {\bf Functions 1:} Assume the sets A = $\{ Stuttgart, Paris, Asuncion, Washington \}$  and  \\ B = $\{ Germany, France, Paraguay, USA \}$ \\ and the relation R = $\{\langle Stuttgart, Germany \rangle, \langle Paris, France \rangle, \langle Asuncion, Paraguay \rangle\}$ in the Cartesian product of A and B. Is R a function from A to B?

\begin{enumerate}[noitemsep]
        \item Yes, because every element of A that is the first element of a pair in R is mapped to exactly one element of B. (0pts)
	\item No, because not every element of A is mapped to an element of B. (2pts)
\end{enumerate}

{\bf Model answer:} (b) is the correct answer. For R to be a function from A to B, every element of A would need to be mapped to a exactly one element of B (see the definition of functions on p.88). R is not a function because  R does not map `Washington'  to an element of B. 

\item {\bf Exercise 12 on p.85:} Which answer is correct? You may only write the sentence ``Because $\rule{1cm}{0.15mm}$ expresses a $\rule{1cm}{0.15mm}$ and $\rule{1cm}{0.15mm}$ does not'', with the three $\rule{1cm}{0.15mm}$ filled in.

{\bf Model answer:} The correct answer is ``Because {\em sibling} expresses a symmetric relation and {\em brother} does not''. To see that {\em brother} is not symmetric, consider Jake and Maggie Gyllenhaal. Jake is Maggie's brother, but not vice versa.

\item {\bf Exercise 13 on p.85:} Which answer is correct? You may only write the sentence ``Because $\rule{1cm}{0.15mm}$ expresses a $\rule{1cm}{0.15mm}$ and $\rule{1cm}{0.15mm}$ does not'', with the three $\rule{1cm}{0.15mm}$ filled in.

{\bf Model answer:} The correct answer is ``Because {\em be to the left of} expresses a transitive relation and {\em be immediately to the left of} does not''. 

\item {\bf Functions 2:} Assume the set P = \{kim, dana, sandy, alex\}, the set H = \{happy, sad\} and the set T = \{1, 0\}. Assume that R1 and R2 are relations from P to H, that R3 is a relation from P to P, that R4 is a relation from P to T and that R5 to R7 are relations from T to T. Which of these relations are functions? Select all that are functions.

\begin{enumerate}[noitemsep]
        \item R1 = \{ $\langle$ kim, happy $\rangle$, $\langle$ dana, happy $\rangle$,  $\langle$ sandy, happy $\rangle$,  $\langle$ alex, happy $\rangle$ \} \\ (2 checked / -2 unchecked)

        \item R2 = \{ $\langle$ kim, sad $\rangle$, $\langle$ dana, happy $\rangle$,  $\langle$ sandy, happy $\rangle$,  $\langle$ alex, happy $\rangle$ \} \\ (2 checked / -2 unchecked)
        
	\item R3 = \{ $\langle$ kim, kim $\rangle$, $\langle$ dana, dana $\rangle$,  $\langle$ sandy, sandy $\rangle$ \} \\ (-2 checked / 2 unchecked)

	 \item R4 = \{ $\langle$ kim, 1 $\rangle$, $\langle$ dana, 1 $\rangle$,  $\langle$ sandy, 0 $\rangle$,  $\langle$ alex, 0 $\rangle$ \} \\ (2 checked / -2 unchecked)
	 
	 \item R5 =  \{ $\langle$ 1, 1 $\rangle$, $\langle$ 1, 1 $\rangle$,  $\langle$ 0, 0 $\rangle$,  $\langle$ 0, 0 $\rangle$ \} \\ (2 checked / -2 unchecked)
	 
 	 \item R6 =  \{ $\langle$ 1, 1 $\rangle$, $\langle$ 1, 0 $\rangle$ \} \\ (-2 checked / 2 unchecked)
	 
  	 \item R7 =  \{ $\langle$ 1, 1 $\rangle$, $\langle$ 0, 1 $\rangle$ \} \\ (2 checked / -2 unchecked)
	 
\end{enumerate}

{\bf Model answer:} The relations R1, R2, R4, R5 and R7 are functions: in each of these, all elements of the domain of the relation are mapped to exactly one element of the codomain of the relation. 

The relations in R3 and R6 are not functions. R3 is not a function because `alex' is an element of the set P but not mapped to any element of the set H. R6 is not a function because the element `1' of T occurs as the first element of two pairs that differ on the second element, i.e., `1' is once mapped to `1' and once to `0'.

\item {\bf Modification of exercise 16 on p.89f.:} Assume the set of all humans as the domain on Monday, May 18, 2020 for the nouns {\em height, age} and {\em citizenship}. Which of these are relational nouns and which of these are functional nouns?

\begin{enumerate}[noitemsep]
        \item {\em height} is a functional noun. (2 checked / -2 unchecked)
        
        \item {\em age} is a functional noun. (2 checked / -2 unchecked)
                
         \item {\em citizenship} is a functional noun.  (-2 checked / 2 unchecked)
         
\end{enumerate}

{\bf Model answer:} Both {\em height} and {\em age} are functional nouns: every human is mapped to a unique height and a unique age. The noun {\em citizenship} is a relational noun: some humans have two citizenships, so it is not the case that every human is mapped to just one element of the set of citizenships.

\item {\bf Exercise 17 on p.91:} Which answer has the correct answers to the three parts (a), (b) and (c) of this question?

\begin{enumerate}[noitemsep]
        \item (a) Bj\"orn (b) false (c) true (3pts)
        
         \item (a) Bj\"orn (b) true (c) true (2pts)
         
          \item (a) Bj\"orn (b) false (c) false (2pts)
          
          \item (a) Agnetha (b) false (c) true (2pts)
        
         \item (a) Agnetha (b) true (c) true (1pt)
         
          \item (a) Agnetha (b) false (c) false (1pt)
         
\end{enumerate}

{\bf Model answer:} Answer (a) is the correct answer. For part (a), the `partner' function maps Agnetha to Bj\"orn, as represented by the pair $\langle$ Agnetha, Bj\"orn $\rangle$. 

For part (b), the `partner' function maps Bj\"orn to Agnetha, not to Frida, so the statement is false. 

For part (c), the `partner' function maps Bj\"orn to Agnetha, so $f(f($Bj\"orn$))$ is $f($Agnetha$)$, which is identical to the expression to the right of the equal sign.

\item {\bf Exercise 18 on pp.91f.:} Which answer has the correct answers to the three parts (a), (b), (c) and (d) of this question? 

\begin{enumerate}[noitemsep]
        \item (a) \{$\langle$ Bj\"orn, 1 $\rangle$, $\langle$ Benny, 1 $\rangle$, $\langle$ Agnetha, 0 $\rangle$, $\langle$ Frida, 0 $\rangle$\} (b) 1 (c)  1 (d) 0 (4pts)
        
     \item (a) \{$\langle$ Bj\"orn, 1 $\rangle$, $\langle$ Benny, 1 $\rangle$ $\rangle$\} (b) 1 (c)  1 (d) 0 (3pts)
     
      \item (a) \{$\langle$ Bj\"orn, 1 $\rangle$, $\langle$ Benny, 1 $\rangle$, $\langle$ Agnetha, 0 $\rangle$, $\langle$ Frida, 0 $\rangle$\} (b) 0  (c)  0 (d) 1  (1pt)
      
       \item (a) \{$\langle$ Bj\"orn, 1 $\rangle$, $\langle$ Benny, 1 $\rangle$ \} (b) 0 (c)  0 (d) 1 (0pts)
         
\end{enumerate}

{\bf Model answer:} Answer (a) is the correct answer. For part (a), the characteristic function of the set of male individuals is the function that maps every male individual in ABBA to 1 and every female individual in ABBA to 0. Answers (b) and (d) are incorrect because the female individuals in ABBA are not mapped to 0. 

For part (b), the value of applying the {\sf male} function to the individual Bj\"orn is 1: remember that the function maps an individual to 1 if the individual is male and to 0 if the individual is female.

For part (c), the value of applying the function denoted by {\em is male}, which is the function defined in part (a), to Bj\"orn is 1.

For part (d), the value of applying the function denoted by {\em is male}, which is the function defined in part (a), to Agnetha is 0.

\end{enumerate}
\end{document}
