\documentclass[a4,11pt]{article}
\usepackage{tikz} %to draw models
\usetikzlibrary{tikzmark}
\usepackage{multicol} %for multi columns
\usepackage[utf8]{inputenc}
\usepackage{gb4e}
\usepackage{enumitem}
\usepackage{amsmath} %for fractions
\usepackage{fullpage}
\usepackage{graphicx}
\graphicspath{ { } }
\usepackage{tipa}


\def\bad{{\leavevmode\llap{*}}}
\def\marginal{{\leavevmode\llap{?}}}
\def\verymarginal{{\leavevmode\llap{??}}}
\def\infelic{{\leavevmode\llap{\#}}}

\setlength{\parindent}{0in}

\definecolor{Pink}{RGB}{240,0,120}
\newcommand{\jt}[1]{\textbf{\textcolor{Pink}{#1}}}

% Semantic brackets
\newcommand{\6}{\mbox{$[\hspace*{-.6mm}[$}} 
\newcommand{\9}{\mbox{$]\hspace*{-.6mm}]$}}
\newcommand{\sem}[2]{\6#1\9$^{#2}$}

\title{Week 10: Quiz questions and model answers}
\author{Judith Tonhauser and Elena Vaik\v snorait\.{e} }
\date{\today}


\begin{document}

\maketitle

{\bf Introductory message:} This quiz covers the material in K 21, except 21.4.2 and 21.5. This chapter delves deeper into tense. 

\begin{enumerate}[leftmargin = 12pt]

\item Match a term to the appropriate definition. (each correct match 1 point)

Term
\begin{enumerate}[noitemsep]
\item Tense
\item Aspect
\end{enumerate}

Definition

\begin{enumerate}[noitemsep]
\item indicates a temporal relation between Topic Time and Time of Utterance
\item indicates a temporal relation between Topic Time and Time of Situation
\end{enumerate}

{\bf Model answer} The correct answers are "tense indicates a temporal relation between Topic Time and Time of Utterance" and "aspect indicates a temporal relation between Topic Time and Time of Situation".

\item Match a term to the appropriate definition. (each correct match 1 point)

Term

\begin{enumerate}[noitemsep]
\item Past tense 
\item Present tense 
\item Future tense
\end{enumerate}

Definition

\begin{enumerate}[noitemsep]
\item Topic Time is prior to Time of Utterance
\item Time of Utterance is prior to Topic Time 
\item Time of Utterance  is contained within Topic Time.
\end{enumerate}


{\bf Model answer} The correct answers as past tense (a), present tense (b) and future tense (c).


\item Which sentence contains the verb play in simple present tense form? (1 point)

\begin{enumerate}[noitemsep]
\item John is playing.
\item John plays.
\end{enumerate}

{\bf Model answer} The correct answer is (b). The verb in (a) is in present progressive form.


\item In which sentence does the verb indicate that the time of utterance is contained within topic time? (2 points)
\begin{enumerate}[noitemsep]
\item John is playing soccer.
\item John plays soccer.
\end{enumerate}

{\bf Model answer} The correct answer is (a). The use of present progressive indicates that John is playing at the time of the utterance. The verb in (b) has a habitual reading.

\item Which generalization is true about the English present tense? (2 points)

\begin{enumerate}[noitemsep]
\item Only states may be described using the simple present tense form of the verb to indicate that TU is contained within TT.
\item Only event may be described using the simple present tense form of the verb to indicate that TU is contained within TT.\end{enumerate}

{\bf Model answer} The correct answer is (a). Events in simple present tense can give rise to habitual readings. For example, the verb \textit{play} is an eventive verb (you can use the diagnostics you learned last week to make sure). Event verb used in simple as in \textit{John plays soccer}, it indicates that John plays soccer regularly.


\item What is the topic time of the bolded verb in the sentence \textit{When I got home from the hospital, my wife {\bf wrote} a letter to my doctor}? (3 points)

\begin{enumerate}[noitemsep]
\item now
\item after with the event described in the temporal clause introduced by when
\item overlapping with the event described in  the temporal clause introduced by when
\end{enumerate}

{\bf Model answer} The correct answer is (b).

\item What is TSit of the bolded verb in the sentence \textit{When I got home from the hospital, my wife {\bf was writing} a letter to my doctor}?  (3 points)

\begin{enumerate}[noitemsep]
\item now
\item after with the event described in the temporal clause introduced by when
\item overlapping with the event described in the temporal clause introduced by when
\end{enumerate}


{\bf Model answer} The correct answer is (c). The use of progressive aspect indicates that the situation which is being described is contained within the topic time.


\end{enumerate}
\end{document}
