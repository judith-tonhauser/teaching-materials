\documentclass[a4,11pt]{article}
\usepackage{tikz} %to draw models
\usetikzlibrary{tikzmark}
\usepackage{multicol} %for multi columns
\usepackage[utf8]{inputenc}
\usepackage{gb4e}
\usepackage{enumitem}
\usepackage{amsmath} %for fractions
\usepackage{fullpage}
\usepackage{graphicx}
\graphicspath{ { } }
\usepackage{tipa}


\def\bad{{\leavevmode\llap{*}}}
\def\marginal{{\leavevmode\llap{?}}}
\def\verymarginal{{\leavevmode\llap{??}}}
\def\infelic{{\leavevmode\llap{\#}}}

\setlength{\parindent}{0in}

\definecolor{Pink}{RGB}{240,0,120}
\newcommand{\jt}[1]{\textbf{\textcolor{Pink}{#1}}}

% Semantic brackets
\newcommand{\6}{\mbox{$[\hspace*{-.6mm}[$}} 
\newcommand{\9}{\mbox{$]\hspace*{-.6mm}]$}}
\newcommand{\sem}[2]{\6#1\9$^{#2}$}

\title{Week 10: Quiz questions and model answers}
\author{Judith Tonhauser and Elena Vaik\v snorait\.{e} }
\date{\today}


\begin{document}

\maketitle

{\bf Introductory message:} This quiz covers the material in K ch.\ 21, except 21.4.2 and 21.5. This chapter delves deeper into tense. 

\begin{enumerate}[leftmargin = 12pt]

\item {\bf Tense vs Aspect} Match each term to the appropriate definition. (each correct match 1 point)

Term
\begin{enumerate}[noitemsep]
\item Tense
\item Aspect
\end{enumerate}

Definition

\begin{enumerate}[noitemsep]
\item indicates a temporal relation between Topic Time and Time of Utterance
\item indicates a temporal relation between Topic Time and Time of Situation
\end{enumerate}

{\bf Model answer} The correct answers are "tense indicates a temporal relation between Topic Time and Time of Utterance" and "aspect indicates a temporal relation between Topic Time and Time of Situation".

\item  {\bf Tenses} Match the term to the appropriate definition. (each correct match 1 point)

Term

\begin{enumerate}[noitemsep]
\item Past tense 
\item Present tense 
\item Future tense
\end{enumerate}

Definition

\begin{enumerate}[noitemsep]
\item Topic Time is prior to Time of Utterance
\item Time of Utterance  is contained within Topic Time.
\item Time of Utterance is prior to Topic Time 
\end{enumerate}

{\bf Model answer} The correct answers as past tense (a), present tense (b) and future tense (c).


\item {\bf Simple present} Which sentence contains the verb {\em play} in the simple present tense form? (1 point)

\begin{enumerate}[noitemsep]
\item John is playing.
\item John plays.
\end{enumerate}

{\bf Model answer} The correct answer is (b). The verb in (a) is in present progressive form.


\item {\bf Habitual reading} Which sentence has a habitual reading? (1 point)
\begin{enumerate}[noitemsep]
\item John is playing soccer.
\item John plays soccer.
\end{enumerate}

{\bf Model answer} The correct answer is (b). The verb \textit{play} is an event verb (you can use the diagnostics you learned last week to make sure). Event verb in simple present can have a habitual reading. In (a), the use of present progressive indicates that John is playing at the time of the utterance. 


\item {\bf Topic Time} What is the topic time of the bolded verb in the sentence \textit{When I got home from the hospital, my dog had fleas}? (3 points)

\begin{enumerate}[noitemsep]
\item  now
\item the time the speaker got home from the hospital
\item the time before the speaker went to the hospital
\end{enumerate}

{\bf Model answer} The correct answer is (b). 



\item {\bf TSit} What is situation time of the bolded verb in the sentence \textit{When I got home from the hospital, my wife {\bf was writing} a letter to my doctor}?You can assume that the progressive captures the same relative locations of the Time of Utterance, Topic Time and Situation time as the imperfective on p.390. (3 points)

\begin{enumerate}[noitemsep]
\item now
\item before the the time the speaker got home from the hospital
\item after  the the time the speaker got home from the hospital
\item overlapping with  the time the speaker got home from the hospital
\end{enumerate}


{\bf Model answer} The correct answer is (d). The use of progressive aspect indicates that the situation which is being described overlaps with the topic time. In other words, the wife started writing the letter before the speaker got home from the hospital and she has not finished writing it when the speaker got home.



\item {\bf What is the tense?} Which of the following sentences feature a past tense finite verb? Select all that apply.

\begin{enumerate}[noitemsep]

\item I was hungry.  (1pt selected / -1pt unselected)

\item Kim has slept in a tent. (-1pt selected / 1pt unselected)

\item Cameron was going to fly to France tomorrow. (1pt selected / -1pt unselected)

\item Taylor had beat the odds. (1pt selected / -1pt unselected)

\item Sam bit the dog. (1pt selected / -1pt unselected)

\end{enumerate}

{\bf Model answer:} The finite verbs in (a), (c), (d) and (e) are past tense: {\em was, was, had} and {\em bit}. The finite verb in (b) is present tense: {\em has}. If you selected (b), be careful to distinguish tense and aspect: the situation of Kim sleeping in a tent is located in the past of the utterance time, but this is not due to the tense but due to the perfect aspect.


\item  {\bf Identify tense} In English there is no special morphology that indicates future. As discussed on p.408-409, present tense morphology can be used to describe events in the future. Therefore it has been argued that from English has only two tenses: past and non-past, whereby non-past can be used to describe events/situations in the present and the future. Based on what you just learned, match each of the following sentences to a tense (past/non-past) based on the morphology of the finite verb? (1 pt each correct match)


\begin{enumerate}[noitemsep]

\item I was given an apple.  

\item Kim loves sleeping on the couch.

\item  The train leaves at 5pm.

\item Lily has been writing a letter since 5am.

\item Tom had bought paint before.

\item Tim bought a bike.

\end{enumerate}

{\bf Model answer:} The verbs in (a), (e) and (f) are past. In (b), (c), (d) the verbs are in non-past. Notice that the verb in (b) has a habitual reading, while the sentence in (c) describes a future event.
\end{enumerate}
\end{document}
