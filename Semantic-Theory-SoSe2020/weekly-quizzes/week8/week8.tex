\documentclass[a4,11pt]{article}
\usepackage{tikz} %to draw models
\usetikzlibrary{tikzmark}
\usepackage{multicol} %for multi columns
\usepackage[utf8]{inputenc}
\usepackage{gb4e}
\usepackage{enumitem}
\usepackage{amsmath} %for fractions
\usepackage{fullpage}
\usepackage{graphicx}
\graphicspath{ { } }


\setlength{\parindent}{0in}

\definecolor{Pink}{RGB}{240,0,120}
\newcommand{\jt}[1]{\textbf{\textcolor{Pink}{#1}}}

% Semantic brackets
\newcommand{\6}{\mbox{$[\hspace*{-.6mm}[$}} 
\newcommand{\9}{\mbox{$]\hspace*{-.6mm}]$}}
\newcommand{\sem}[2]{\6#1\9$^{#2}$}

\title{Week 7: Quiz questions and model answers}
\author{Judith Tonhauser and Elena Vaik\v snorait\.{e} }
\date{\today}


\begin{document}

\maketitle

{\bf Introductory message:} 

\begin{enumerate}[leftmargin = 12pt]

\item {\bf  Material implication: Ex.10 p. 100.} Specify the semantics rule for the material conditional (correct answer 2 points, incorrect answer -2 points).

\begin{enumerate}
\item If $\phi$ and $\psi$ are formulae, then $[[\phi \to \psi]]^{I}$ = 1 if $[[\phi]]^{I}$ = 0 or if $[[\phi]]^{I}$ = 1 and $[[\psi]]^{I}$ = 1, and 0 otherwise. 
\item If $\phi$ and $\psi$ are formulae, then $[[\phi \to \psi]]^{I}$ = 1 if $[[\phi]]^{I}$ = 1 and $[[\psi]]^{I}$ = 1, and 0 otherwise. 
\item If $\phi$ and $\psi$ are formulae, then $[[\phi \iff \psi]]^{I}$ = 1 if $[[\phi]]^{I}$ = 0 or if $[[\phi]]^{I}$ = 1 and $[[\psi]]^{I}$ = 1, and 0 otherwise. 
\item If $\phi$ and $\psi$ are formulae,  then $[[\phi \to \psi]]^{I}$ = 1 if $[[\psi]]^{I}$ = 1, and 0 otherwise. 
\end{enumerate}

{ \bf Model answer:}  The correct answer is (a). As shown in the table on p.99, the material implication is true when the antecedent is false (regardless of the truth value of the consequent) and when both the antecedent and the consequent are true.

 
\item {\bf Validity} Using on the truth table below, determine whether the argument of the form \textit{ P $\to$ R; R: therefore $\neg$ P} is valid (correct answer 1 point, incorrect answer -1 point).


\begin{tabular}{c | c | c| c }
\hline \hline
P & R & P $\to$ R  & $\neg$ P \\
\hline
1 & 1 & 1 & 0 \\
1 & 0 & 0 & 0 \\
0 & 1 & 1 & 1 \\
0 & 0 & 1 & 1 \\
\hline \hline
\end{tabular}

\begin{enumerate}
\item The argument is valid, because when the conclusion is true in every case that the premises are true.
\item The argument is not valid, because it is not the case that the conclusion is true in every case that the premises are true.
\end{enumerate}

{ \bf Model answer:}  The correct answer is (b), the argument is not valid. It is not valid because there is one case in which the premises are true (R is true and P $\to$ R is true) but the conclusion is false.

\item {\bf Tautology: ex.17 p.105} . Using truth tables, check whether the following pairs of formulae are tautologies. Select all tautologic formulae.

\begin{enumerate}
\item $[P \lor Q]$
\item $[[P\rightarrow Q]  \lor  [Q \rightarrow P]]$
\item $[[P \rightarrow Q] \iff \neg Q \lor \neg P ]$ 
\item $[[[P \lor Q] \rightarrow  R] \iff [[P \rightarrow Q] \land [P \rightarrow  Q]]]$
\end{enumerate}


{ \bf Model answer:} The correct answer is (b). As shown in the truth table below, the formula is true under every assignment.

\begin{tabular}{c | c | c | c | c}
\hline \hline
P & Q &  ( P$\rightarrow$Q) & ( Q $\rightarrow$ P )  & ( P$\rightarrow$Q) $\lor$ ( Q $\rightarrow$ P ) \\
\hline
T & T & T& T & T\\
T & F & F& T & T\\
F & T & T & F& T\\
F & F & T& T& T \\
\hline \hline
\end{tabular}

\end{enumerate}
\end{document}
