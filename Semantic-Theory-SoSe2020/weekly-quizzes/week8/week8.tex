\documentclass[a4,11pt]{article}
\usepackage{tikz} %to draw models
\usetikzlibrary{tikzmark}
\usepackage{multicol} %for multi columns
\usepackage[utf8]{inputenc}
\usepackage{gb4e}
\usepackage{enumitem}
\usepackage{amsmath} %for fractions
\usepackage{fullpage}
\usepackage{graphicx}
\graphicspath{ { } }
\usepackage{tipa}


\setlength{\parindent}{0in}

\definecolor{Pink}{RGB}{240,0,120}
\newcommand{\jt}[1]{\textbf{\textcolor{Pink}{#1}}}

% Semantic brackets
\newcommand{\6}{\mbox{$[\hspace*{-.6mm}[$}} 
\newcommand{\9}{\mbox{$]\hspace*{-.6mm}]$}}
\newcommand{\sem}[2]{\6#1\9$^{#2}$}

\title{Week 9: Quiz questions and model answers}
\author{Judith Tonhauser and Elena Vaik\v snorait\.{e} }
\date{\today}


\begin{document}

\maketitle

{\bf Introductory message:} This quiz covers the material in CC chs. 3.2.3 to 3.3.3.

\begin{enumerate}[leftmargin = 12pt]

\item {\bf How are the symbols pronounced?} The textbook uses the Greek letter symbols $\phi$ and $\psi$. How are they pronounced? Match the symbols to the pronunciations. (1pt for each match)

Greek letter symbols:

\begin{enumerate}
\item $\phi$
\item $\psi$
\end{enumerate}

Pronunciations:

\begin{enumerate}
\item {} [\textipa{fa\i}]
\item {} [\textipa{sa\i}]
\end{enumerate}

{\bf Model answer:} $\phi$ is pronounced [\textipa{fa\i}] and $\psi$ is pronounced [\textipa{sa\i}]. Note that there is no \textipa{p} in these pronunciations!

\item {\bf  Antecedent and consequent 1} In the English conditional sentence {\em If Trump were Captain of the RMs Titanic, he would say that there isn't an iceberg}, which sentence is the antecedent and which is the consequent? (1pt for each match)

Terminology:

\begin{enumerate}
\item antecedent
\item consequent
\end{enumerate}

Sentences:

\begin{enumerate}
\item Trump were Captain of the RMs Titanic
\item he would say that there isn't an iceberg
\end{enumerate}

{ \bf Model answer:}  The sentence in (a) is the antecedent of the conditional and the sentence (b) is the consequent of the conditional. The sentence that begins with {\em if} is the antecedent.

\item {\bf  Antecedent and consequent 2} In the English conditional sentence {\em I will eat my hat if the restaurant is closed}, which sentence is the antecedent and which is the consequent? (1pt for each match)

Terminology:

\begin{enumerate}
\item antecedent
\item consequent
\end{enumerate}

Sentences:


\begin{enumerate}
\item the restaurant is closed
\item I will eat my hat
\end{enumerate}

{ \bf Model answer:}  The sentence in (a) is the antecedent of the conditional and the sentence (b) is the consequent of the conditional. The sentence that begins with {\em if} is the antecedent.

\item {\bf  Antecedent and consequent 3} In the logic sentence $p \rightarrow q$, which sentence is the antecedent and which is the consequent? (1pt for each match)

Terminology:

\begin{enumerate}
\item antecedent
\item consequent
\end{enumerate}

Sentences:

\begin{enumerate}
\item p
\item q
\end{enumerate}

{ \bf Model answer:}  The sentence in (a) is the antecedent of the conditional and the sentence (b) is the consequent. The sentence before the material conditional  is the antecedent.

\item {\bf  Antecedent and consequent 4} In the logic sentence $ q \rightarrow (p \wedge r)$, which sentence is the antecedent and which is the consequent? (1pt for each match)

Terminology:

\begin{enumerate}
\item antecedent
\item consequent
\end{enumerate}

Sentences:

\begin{enumerate}
\item $ q$
\item $p \wedge r$
\end{enumerate}

{ \bf Model answer:}  The sentence in (a) is the antecedent of the conditional and the sentence (b) is the consequent. The sentence before the material conditional  is the antecedent.

\item {\bf  Antecedent and consequent 5} In the logic sentence $(r \vee u) \rightarrow p$, which sentence is the antecedent and which is the consequent? (1pt for each match)

Terminology:

\begin{enumerate}
\item antecedent
\item consequent
\end{enumerate}

Sentences:

\begin{enumerate}
\item $r \vee u$
\item $p$
\end{enumerate}

{ \bf Model answer:}  The sentence in (a) is the antecedent of the conditional and the sentence (b) is the consequent. The sentence before the material conditional  is the antecedent.

\item {\bf  Material conditional} The material conditional ($\rightarrow$) is one way in which English conditionals are translated into propositional logic. As noted on p.110, $p \rightarrow q$ is false only if $p$ is true and $q$ is false. Thus, when $p$ is false, $p \rightarrow q$ is true regardless of whether $q$ is true or false. For some English conditional sentences, this makes intuitive sense, for others it doesn't. As discussed on p.111, one sentence for which it doesn't make so much sense is the English conditional sentence {\em If the moon is made of green cheese, then I had yoghurt for breakfast}. Why does it not make sense to analyze this English conditional sentence using the material conditional? 

Answer: If the English conditional sentence  {\em If the moon is made of green cheese, then I had yoghurt for breakfast} is analyzed using the material conditional, \ldots

\begin{enumerate}
\item \ldots then the conditional sentence is predicted to be true, because the antecedent is false. That does not make sense because native speakers would not judge the English conditional sentence to be true. (3 pts)

\item \ldots then the consequent is predicted to be true, because the antecedent is false. That does not make sense because the consequent does not have to be true.  (0 pts)

\end{enumerate}

{ \bf Model answer:}  Answer (a) is correct. If the English conditional sentence  {\em If the moon is made of green cheese, then I had yoghurt for breakfast} is analyzed using the material conditional, then the conditional sentence is predicted to be true because the antecedent is false, and that does not accord with native speaker intuitions. Answer (b) is incorrect because the consequent is not predicted to be true; rather, the consequent can be true or false if the antecedent is false.

\item {\bf  In words} How do we say the logic sentences in English? Match the logic sentences and the English sentences. (2pts for each match)

Logic sentences

\begin{enumerate}
\item $p \rightarrow (q \vee r)$

\item $p \rightarrow (q \vee \neg r)$

\item $p \vee (q \rightarrow r)$

\item $p \vee (\neg q \rightarrow r)$

\end{enumerate}

English sentences

\begin{enumerate}
\item if p then q or r

\item if p then q or not r

\item p or if q then r

\item p or if not q then r

\end{enumerate}

{ \bf Model answer:} (a) matches (a), (b) matches (b), (c) matches (c) and (d) matches (d). Note that the English sentences do not necessarily disambiguate the logic sentences: for instance, an utterance of ``if p then q or r'' would be the way to say both $p \rightarrow (q \vee r)$ and $(p \rightarrow q) \vee r$. If we really want to disambiguate the logic sentences, we would have to also mention the parentheses, e.g., ``if p then open parentheses q or r closed parentheses''.

\item {\bf  Exercise 10 on p.112} Do Exercise 10 on p.112. (max.\ 400 characters; 6 points)

{ \bf Model answer:}  The syntactic and semantic rules of other two-place connectives of propositional logic were given in the previous sections. We can model our answer based on them, e.g., the syntactic rule for conjunction (p.102) and the semantic rule for conjunction (p.103). For the semantic rule, we use the truth table on p.110. 

{\bf Syntactic rule: Material conditional}
\\ If $\phi$ and $\psi$ are formulas, then [$\phi \rightarrow \psi$] is also a formula. 
\\ In words: ``If phi and psi are formulas, then phi rightarrow psi (or: then if phi then psi) is also a formula.''

{\bf Semantic rule: Material conditional}
\\ If $\phi$ and $\psi$ are formulas, then \sem{$\phi \rightarrow \psi$}{I} = 0 if \sem{$\phi$}{I} = 1 and \sem{$\psi$}{I} = 0, and 1 otherwise.
\\ In words: ``If phi and psi are formulas, then if phi then psi is false under interpretation I if phi under interpretation I is true and psi under interpretation I is false, and if phi then psi is true otherwise.''


\item {\bf  Exercise 12 on p.112f., part 1} Fill out the truth table. To give your answer, match the truth values for P and Q (first two columns) to the truth values for the last three columns: (2 pts for each correct match)

Truth values for P and Q (in that order):

\begin{enumerate}
\item 1, 1
\item 1, 0
\item 0, 1
\item 0, 0
\end{enumerate}

Rows ([P $\rightarrow$ Q], $\neg$ Q, $\neg$ P):

\begin{enumerate}
\item 1, 0, 0
\item 0, 1, 0
\item 1, 0, 1
\item 1, 1, 1
\end{enumerate}

{ \bf Model answer:}  (a) matches (a), (b) matches (b), (c) matches (c) and (d) matches (d). So the final truth table looks as follows:

\begin{tabular}{ccccc}
P & Q & [P $\rightarrow$ Q] & $\neg$Q & $\neg$P \\ \hline
1 & 1 & 1 & 0 & 0 \\ 
1 & 0 & 0 & 1 & 0 \\
0 & 1 & 1 & 0 & 1 \\
0 & 0 & 1 & 1 & 1 \\
\end{tabular}

\item {\bf  Exercise 12 on p.112f., part 2} Explain in your own words how we can see from the truth table above that Modus Tollens is valid. (max.\ 350 characters; 6 points)

{ \bf Model answer:}  An argument is valid if and only if if the premises are true, the conclusion is true. In Modus Tollens, the premises are [P $\rightarrow$ Q] and $\neg$Q. We can see in the fourth row of the truth table that if both of these premises are true (1), then the conclusion, $\neg$P, is also true (1). Thus, Modus Tollens is valid. (324 characters)

\item {\bf  Exercise 13 on p.113} Do Exercise 13 on p.113.  (max.\ 350 characters; 6 points)

{ \bf Model answer:}  From the truth table for Denying the Antecedent, given below, we see that if the premises (which are [P $\rightarrow$ Q] and $\neg$P) are true (1), then the conclusion ($\neg$Q) is not necessarily true: it is false (0) in row 3 and true (1) in row 4. Thus, Denying the Antecedent is not a valid argument.

\begin{tabular}{ccccc}
P & Q & [P $\rightarrow$ Q] & $\neg$P & $\neg$Q \\ \hline
1 & 1 & 1 & 0 & 0 \\ 
1 & 0 & 0 & 0 & 1 \\
0 & 1 & 1 & 1 & 0 \\
0 & 0 & 1 & 1 & 1 \\
\end{tabular}

\item {\bf  Calculating truth with the biconditional} Assume the following propositional logic formula: $a \wedge (b \leftrightarrow c)$. Match the truth values for $a, b$ and $c$ to the truth values for the entire formula. (3pts for each match)

Truth values for $a, b$ and $c$ (in that order):

\begin{enumerate}
\item 1, 1, 0

\item 1, 0, 0

\end{enumerate}

Truth value for the entire formula:

\begin{enumerate}
\item 1

\item 0

\end{enumerate}

{ \bf Model answer:}  Here is the truth table for the formula. (This was not required, but you might want to practice building such truth tables anyway.) Note that it has 8 rows because there are 8 possible truth value combinations for $a, b$ and $c$ (as shown in the first three columns).

\begin{tabular}{ccccc}
a & b & c & $[b \leftrightarrow c]$ & $a \wedge [b \leftrightarrow c]$ \\ \hline
1 & 1 & 1 & 1 &  1 \\ 
1 & 1 & 0 & 0 &  0 \\
1 & 0 & 1 & 0 &  0 \\
1 & 0 & 0 & 1 &  1 \\
0 & 1 & 1 & 1 &  0 \\ 
0 & 1 & 0 & 0 &  0 \\
0 & 0 & 1 & 0 &  0\\
0 & 0 & 0 & 1 &  0\\
\end{tabular}

The truth values $a, b$ and $c$ in (a) above match the second row. In that case, the entire formula is false (0). The truth values $a, b$ and $c$ in (b) above match the fourth row. In that case, the entire formula is true (1).

\item {\bf  Tautology} Is $a \wedge [b \leftrightarrow c]$ (see the previous question) a tautology?

\begin{enumerate}
\item yes (0 pts)
\item no (2 pts)
\end{enumerate}

{ \bf Model answer:}  A formula is a tautology if the formula is true under every assignment of truth values to the atomic sentences. $a \wedge [b \leftrightarrow c]$ is not a tautology because there is at least one assignment of truth values of $a, b$ and $c$ under which the formula is not true, as shown in the answer to the previous question.

\end{enumerate}
\end{document}
