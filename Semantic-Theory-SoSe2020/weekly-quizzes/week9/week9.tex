\documentclass[a4,11pt]{article}
\usepackage{tikz} %to draw models
\usetikzlibrary{tikzmark}
\usepackage{multicol} %for multi columns
\usepackage[utf8]{inputenc}
\usepackage{gb4e}
\usepackage{enumitem}
\usepackage{amsmath} %for fractions
\usepackage{fullpage}
\usepackage{graphicx}
\graphicspath{ { } }
\usepackage{tipa}


\setlength{\parindent}{0in}

\definecolor{Pink}{RGB}{240,0,120}
\newcommand{\jt}[1]{\textbf{\textcolor{Pink}{#1}}}

% Semantic brackets
\newcommand{\6}{\mbox{$[\hspace*{-.6mm}[$}} 
\newcommand{\9}{\mbox{$]\hspace*{-.6mm}]$}}
\newcommand{\sem}[2]{\6#1\9$^{#2}$}

\title{Week 9: Quiz questions and model answers}
\author{Judith Tonhauser and Elena Vaik\v snorait\.{e} }
\date{\today}


\begin{document}

\maketitle

{\bf Introductory message:} This quiz covers the material in K chs.\ 8.1-8.3 and ch.\ 9. These chapters review concepts like entailment, presupposition and conversational implicature, which were already introduced in week 2, but also go deeper into the topic of conversational implicatures.

\begin{enumerate}[leftmargin = 12pt]

\item { \bf Sentence vs. utterance} A sentence is a grammatical unit that has syntactic structure and expresses some idea. It is an abstract entity. An utterance corresponds to an actual event whereby a sentence is produced. Match each property as belonging to an utterance or to a sentence. (each correct match 1pt)

\begin{enumerate}[noitemsep]
\item volume
\item truth value
\item location
\item proposition
\item speed
\item time
\end{enumerate}

{ \bf Model answer:} The correct answers are utterances have (a), (c), (e), (f) properties, while sentences have properties (b) and (d). Since an utterance is an event we can talk about its volume (how loud the speaker was talking at the time), location (where it was uttered), speed (how quickly the speaker was talking) and time (when the speaker produced it). Sentences denote propositions and have truth values.



\item {\bf Sentence meaning versus utterance meaning}  Consider the following exchange:

\begin{exe}
\exi{A:} Is Andreas dating anyone these days?
\exi{B:} Well, he goes to Heidelberg every weekend.
\end{exe}

Considering B, which meaning is the sentence meaning and which is the utterance meaning? (2 pts for each correct match)

Types of meaning:

\begin{enumerate}[noitemsep]
\item sentence meaning
\item utterance meaning
\end{enumerate}

Meanings:

\begin{enumerate}[noitemsep]
\item Andreas goes to Heidelberg every weekend
\item Andreas goes to Heidelberg every weekend and he is dating somebody in Heidelberg
\item Andreas is dating somebody in Heidelberg
\end{enumerate}

{\bf Model answer:} The sentence meaning is `Andreas goes to Heidelberg every weekend'; this is the semantic content of B. The utterance meaning is, as per p.139, ``the totality of what the speaker intends to convey by making an utterance'' and, as per p.143, ``composed of the sentence meaning plus any pragmatic inference created by the specific context of use''. Thus, the utterance meaning is `Andreas goes to Heidelberg every weekend and he is dating somebody in Heidelberg'.



\item { \bf Review: implication} In week 2, we learned different diagnostics that allow us to differentiate between three types of implication. Match each trait to the implication that possesses it (each correct match 1pt)

Types of implication:
\begin{enumerate}[noitemsep]
\item Implicature
\item Entailment
\item Presupposition
\end{enumerate}

Traits:
\begin{enumerate}[noitemsep]
\item Defeasible
\item Reinforceable
\item Projects over entailment-cancelling operators
\item Non-defeasible
\item Does not project over entailment-cancelling operators
\end{enumerate}

{ \bf Model answer:} As per CC p.28, implicatures are defeasible and reinforceable. Entailments are not defeasible nor reinforceable and they do not project over entailment-cancelling operators. Presuppositions are not defeasible nor reinforceable but they do project over entailment-cancelling operators. 

\item  { \bf Which type of implication?} This exercise is a simplified version of the exercise on p.157: For each of the sentences in (a) - (d), i.e., what is given after STATED, determine whether the implication (i.e., what is given after INFERRED) is a conversational implicature, a presupposition or an entailment. (Note: the textbook additionally distinguishes particularized and generalized conversational implicatures, but we ignore this difference in this course.) (each correct match 4pts)


\begin{enumerate}[noitemsep]
\item stated: My mother is the mayor of Waxahachie. \\ 
	inferred: The mayor of Waxahachie is a woman.
\item stated:  That man is either Martha’s brother or her boyfriend.  \\
	inferred: The speaker does not know whether the man is Martha’s brother or boyfriend.
\item stated:  My great-grandfather was arrested this morning for drag racing. \\  
	inferred:  I have a great-grandfather.
\item stated: That’s a great joke – Ham, Shem and Japheth couldn’t stop laughing when they heard it from Noah. \\  
	inferred: The joke has lost some of its freshness.
\end{enumerate}


{ \bf Model answer:} The correct answers are (a) is an entailment, (b) and (d) is a conversational implicature; (c) is a presupposition. 

(a) is an entailment because it is non-defeasible and it does not project. It is non-defeasible because the example {\it My mother is the mayor of Waxahachie. \#In fact, the mayor of Waxahachie is not a woman.} is self-contradictory. It does not project because the sentences  {\it My mother is not the mayor of Waxahachie. Is my mother the mayor of Waxahachie?} do not imply that the mayor of Waxahachie is a woman.

(c) is a presupposition because it is non-defeasible and it projects.  It is non-defeasible because the example {\it  My great-grandfather was arrested this morning for drag racing. \#In fact, I don't have a great-grandfather.} is self-contradictory. It projects because the sentences  {\it My great-grandfather was not arrested this morning for drag racing. Was my great-grandfather arrested this morning for drag racing?} do not imply that the speaker has a a great-grandfather.

(b) and (d) are conversational implicatures because they are defeasible. The examples {\it That man is either Martha’s brother or her boyfriend. In fact, he is her boyfriend.} and {\it That’s a great joke – Ham, Shem and Japheth couldn’t stop laughing when they heard it from Noah. Actually, they haven't heard the joke before.} are not contradictory.


\item  { \bf Identifying maxims 1}  What maxim is violated in B's reply? (2pts)

\begin{exe}
\exi{A:} What is the street address of Elena Vaik\v{s}norait\.{e}'s office?
\exi{B:} Keplerstrasse 17. You take the elevator to the fourth floor, turn right (but you could also turn left) and take the stairs leading up. If you go down, you’ll enter the literature department, and I don’t think you want that, since I assume your business is linguistics related.
\end{exe}

\begin{enumerate}[noitemsep]
\item relevance
\item quantity
\item quality
\item manner
\end{enumerate}

{ \bf Model answer:} The  correct answer is the maxim of quantity, because the contribution is more informative than necessary. 

\item  { \bf Identifying maxims 2}  What maxim is violated in B's reply? (2pts)


\begin{exe}
\exi{A:} What did Leslie do today?
\exi{B:} Leslie read fifty pages and opened her book.
\end{exe}

\begin{enumerate}[noitemsep]
\item relevance
\item quantity
\item quality
\item manner
\end{enumerate}

{ \bf Model answer:} The  correct answer is the maxim of manner, since the contribution is not orderly: you need to open the book to be able to read it.

\item  { \bf Identifying maxims 3}  Instead of saying “Be brief”, Grice's third maxim of manner is "Be brief (avoid unnecessary prolixity)." By phrasing the maxim this way, which two of his maxims of manner did Grice violate?

\begin{enumerate}
\item Avoid obscurity of expression. (2pts selected / -2pts unselected)
\item Avoid ambiguity. (-2pts selected / 2pts unselected)
\item  Be brief (avoid unnecessary prolixity). (2pts selected / -2pts unselected)
\item Be orderly. (-2pts selected / 2pts unselected)
\end{enumerate}

{ \bf Model answer:} The correct answers are "Avoid obscurity of expression" and  "Be brief". 


\item  { \bf Implicature}  What is the likely implicature carried by B's reply? State which maxim(s) (quality, quantity, relevance, manner) is most important in triggering the implicature. Check all that apply. (5 points)

\begin{exe}
\exi{A:} Is Andreas dating anyone these days?
\exi{B:} Well, he goes to Heidelberg every weekend.
\end{exe}

\begin{enumerate}
\item The likely implicature is Andreas is dating somebody in Heidelberg.
\item The likely implicature is Andreas goes to Heidelberg every weekend.
\item The maxim of relevance
\item The maxim of quantity
\item The maxim of  quality
\item The maxim of manner
\end{enumerate}


{ \bf Model answer:} The most likely implicature is that Andreas is dating somebody in Heidelberg. The implicature is triggered by the violations / flouting of maxims of quantity and relevance. The maxim of quantity is flouted/violated because A's question requires a \textit{yes} or \textit{no} response, B's response is therefore not as informative as is required. The maxim of relevance is flouted/violated because B's response seems to be irrelevant to the question asked.


\item { \bf  XOR} Finish the truth table for the exclusive \textit{or} (XOR): (4 points)


Truth values of p and q (in that order)
\begin{enumerate}[noitemsep]
\item T, T
\item T, F
\item F, T 
\item F, F
\end{enumerate}

Truth value of p XOR q
\begin{enumerate}[noitemsep]
\item T
\item F
\end{enumerate}

{ \bf Model answer:} The exclusive disjunction is true when only one of the disjuncts is true. It is false otherwise. \\
\begin{tabular}{ccc}
p & q &  p XOR q \\
\hline
 T & T & F \\
T & F & T \\
 F & T  & T \\
F & F & F
\end{tabular}


\item {\bf Exclusive or inclusive {\em or}?} Which of the examples in (5) on p.163 feature inclusive or and which feature exclusive {\em or}? Select only those that feature inclusive {\em or}. Note that some of these are quite tricky! 

\begin{enumerate}[noitemsep]
\item[(5a)] (2pts selected / -2pts unselected)
\item[(5b)] (-2pts selected / 2pts unselected)
\item[(5c)] (2pts selected / -2pts unselected)
\item[(5d)] (-2pts selected / 2pts unselected)
\end{enumerate}

{\bf Model answer:} The {\em or} in (5a) is inclusive because, presumably, the scholarship could be awarded to a student who had both Swedish and Norwegian ancestry. The {\em or} in (5b) is exclusive: the meaning of (5b) does not change if we continue it with {\em ...but you can't take both the bus and the train and still arrive at 5pm}. The {\em or} in (5c) is inclusive: the speakers can, presumably, refuse a planning permission if the site is both in a particularly sensitive area and there are safety considerations. The {\em or} in (5d) is exclusive: if the addressee stops (i.e., fulfills the command of the first disjunct), the speaker will, presumably, not shoot, i.e., the fulfill the second disjunct. Another way to see that this {\em or} is disjunctive is that one can continue (5d) with {\em ...but both won't happen!} and paraphrase it as {\em Either you stop or I will shoot, but both won't happen!}. 


\item { \bf  Material implication 1:} Which one of the sentences in (6) on p. 163 is false? (2 points)

\begin{enumerate}[noitemsep]
\item (6a)
\item  (6b)
\item  (6c)
\item  (6d)
\end{enumerate}

{ \bf Model answer:}  The correct answer is (c). Material implication is false only when the antecedent is true and the consequent is false. In (c), the antecedent is true (triangles have three sides), but the consequent is false (the moon is not made of green cheese). Therefore, the entire sentence is false.

\item { \bf  Material implication 2} On p. 163, Kroeger argues that analyzing English conditionals as material implication leads to  unexpected inferences. The conditional {\it If you’re hungry, there’s some pizza in the fridge} if treated as a material implication predicts the inference that {\it If there’s no pizza in the fridge, then you’re not hungry.}. Why does this inference arise? (3 points)

\begin{enumerate}
\item The inference arises because of Modus Tollens which allows us to make inferences of the following form {\it If p, then q. Not q. Therefore not p}. Substituting p with \textit{you’re hungry} and q with \textit{there’s some pizza in the fridge}, we derive the inference that {\it If there’s no pizza in the fridge, then you’re not hungry.}
\item The inference arises because of Modus Tollens which allows us to make inferences of the following form {\it If p, then q. Not q. Therefore not p}. Substituting q with \textit{you’re hungry} and p with \textit{there’s some pizza in the fridge}, we derive the inference that {\it If there’s no pizza in the fridge, then you’re not hungry.}
\item The inference arises because of Modus Ponens which allows us to make the following inference {\it If p, then q. p. Therefore q.} Substituting p with \textit{you’re hungry} and q with \textit{there’s some pizza in the fridge}, we derive the inference that {\it If there’s no pizza in the fridge, then you’re not hungry.}
\item The inference arises because of Modus Ponens which allows us to make the following inference {\it If p, then q. p. Therefore q.} Substituting q with \textit{you’re hungry} and p with \textit{there’s some pizza in the fridge}, we derive the inference that {\it If there’s no pizza in the fridge, then you’re not hungry.}
\end{enumerate}

{ \bf Model answer:} The correct answer is (a).

\item { \bf  Quantifier strength} Put the following sentences in the right order. Start with the sentence which contains the strongest quantifier, whereby strength is defined in terms of entailments. Here, the sentence that contains the strongest quantifier entails sentences with other quantifiers. The weakest quantifier does not entail any other sentences. (6 points)

\begin{enumerate}[noitemsep]
\item All dogs bark at mailmen.
\item Many dogs bark at mailmen.
\item Some dogs bark at mailmen.
\end{enumerate}

{ \bf Model answer:} The correct order is (a), (b), (c). \textit{All} is the strongest quantifier because sentence (a) it entails (b) and (c). \textit{Many} is weaker than \textit{all} because sentence (b) does not entail (a). \textit{Some} is the weakest because sentence (c) does not entail sentences (a) or (b).

\item { \bf  Scalar implicature} By uttering {\it Some of the students passed the exam} the speaker implies that not all students passed the exam. Explain how the implicature is derived. (6 points, max 500 characters)

{ \bf Model answer:}  The implicature is derived via pragmatic reasoning. The maxims of quality state that the speaker should try to make true contributions. The fact that the speaker used a weaker expression (\textit{some}) rather than a stronger alternative (\textit{all}) means that the speaker has no evidence for the stronger alternative. The hearer derives the implicature that the speaker does not know that the stronger alternative is true; in other words, that the speaker does not know that all students passed the exam. (490 characters)

\end{enumerate}
\end{document}
