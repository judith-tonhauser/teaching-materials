
\documentclass[11pt,fleqn,a4]{article}
\usepackage{longtable}
\usepackage{enumitem}

\usepackage[margin=2cm]{geometry}
\usepackage[dvips]{graphics}
\usepackage{hyperref}
\usepackage{txfonts}
%\usepackage[spanish]{babel}
%\usepackage{linguex}
\usepackage{amsfonts}
\usepackage{natbib}
\usepackage{gb4e}
\usepackage[all]{xy}
\usepackage{multirow}

\setlength{\parindent}{0in}
\setlength{\parskip}{1ex}
\setlength{\headsep}{0in}

\setlength{\bibsep}{0mm}
\bibpunct[:]{(}{)}{;}{a}{,}{,}

\newcommand{\yi}{\'{\symbol{16}}}
\newcommand{\nasi}{\~{\symbol{16}}}
\newcommand{\hina}{h\nasi na}
\newcommand{\ina}{\nasi na}
%\renewcommand{\abut}{$\supset$\hspace*{-0.07cm}$\subset$}
\newcommand{\tto}{t$_{top}$}
\newcommand{\wtop}{w$_{top}$}
\newcommand{\tc}{t$_c$}
\newcommand{\schwa}{\begin{sideways}e\end{sideways}}

% Semantic brackets
%\newcommand{\iss}[1]{\mbox{\protect\tiny \mbox{#1}}}
%\newcommand{\sem}[2]{\6#1\9$_\iss{#2}$} David's original
\newcommand{\6}{\mbox{$[\hspace*{-.6mm}[$}} 
\newcommand{\9}{\mbox{$]\hspace*{-.6mm}]$}}
\newcommand{\sem}[2]{\6#1\9$_{#2}$}
\newcommand{\jl}{\langle}
\newcommand{\jr}{\rangle}
\newcommand{\ibr}{-\hspace*{-0.1cm}i}
\newcommand{\hbr}{-\hspace*{-0.1cm}h}


\def\bad{{\leavevmode\llap{*}}}
\def\marginal{{\leavevmode\llap{?}}}
\def\verymarginal{{\leavevmode\llap{??}}}
\def\infelic{{\leavevmode\llap{\#}}}

\begin{document}

Sommersemester 2020 \hfill Linguistik/Anglistik, University of Stuttgart

\begin{center}

{\bf \large Semantic Theory}

\end{center}

\vspace*{-.3cm}

{\bf Instructors:} 

\begin{tabular}{l | ll}
& Dr.\ Judith Tonhauser & Elena Vaiksnoraite \\
\hline
Office: & 4.049 K2 &  \\
Office hours: & Fridays 1-2pm &  \\ 
Email: & {\em judith.tonhauser@ling.uni-stuttgart.de} & \\
\hline
\end{tabular}

\medskip

Department of English linguistics front office: Ralf Bothner, 4.057 K2, Tel.\ +49 711 685-83120

{\bf Goals of the course:} The meanings conveyed by natural language
utterances are subtle and complex, arising as a function of the
conventional content of the expressions uttered and of various contextual
factors. Semantics is the study of the conventional system itself, while
pragmatics studies how context plays a role in conveying meaning. This
course is an introduction to the study of meaning, i.e., semantics and
pragmatics. 

{\bf Course schedule:} 

\begin{longtable}{l l}
{\bf Date} & {\bf Topic}  \\ \hline

09.04.19 & Course overview, syllabus   \\ 

\multicolumn{2}{l}{{\bf 1.\ Sentence types, speech acts and meaning types}}  \\

\multicolumn{2}{l}{(\citealt[chs.2,5]{birner2013}; \citealt[ch.4]{ccmg90})} \\

16.04.19 & Sentence types and speech acts \\

23.04.19 & Entailments and conversational implicatures \\

30.04.19 & Presuppositions and other projective content \\

\multicolumn{2}{l}{{\bf 2.\ Noun phrase meanings and negative polarity items}} \\
\multicolumn{2}{l}{(\citealt{ladusaw80}; \citealt[ch.4]{kearns00})} \\ 
 
07.05.19 & Definite and indefinite noun phrases  \\

14.05.19 & Quantificational noun phrases  \\

21.05.19 & Negative polarity items  \\ 

\multicolumn{2}{l}{{\bf 3.\ Verbs: Tense and aspect}} \\
\multicolumn{2}{l}{(\citealt[chs.7,9]{kearns00})} \\ 

28.05.19 & Tense   \\

04.06.19 & Aspect  \\

11.06.19 & {\em No class: Pfingstferien}  \\

\multicolumn{2}{l}{{\bf 4.\ Truth conditions and compositionality}} \\
\multicolumn{2}{l}{(\citealt[ch.2]{gamut})} \\ 

18.06.19 & Truth conditions, the principle of compositionality  \\

25.06.19 & Propositional logic  \\

02.07.19 & Comparing English to propositional logic  \\

09.07.19 & Spill-over class, review session  \\

16.07.19 & In-class exam (cumulative, open book, open notes)   \\ 

\hline

\end{longtable}

{\bf Readings:} 

{\bf Slides:} 

{\bf Important:} Each meeting will consist of me introducing new material and us doing exercises that will help you better understand the material covered. It is therefore vital that you attend class and regularly participate. If you have to miss class, please review the slides and to work with your class mates to understand what was covered.

\newpage

{\bf Assessment:}

\vspace*{-.3cm}

\begin{enumerate}[leftmargin=3ex]

\item {\bf  Homework assignments} (ungraded VL): There will be three short ungraded homework assignments, to help prepare you for the final exam. You will upload your assignments to a folder on ILIAS and we will discuss the answers in class. 

\item {\bf In-class final exam}: In the last class meeting, there will be a cumulative final exam (open book, open notes). The term `open book' means that you can bring the assigned readings; the term `open notes' means that you can bring your notes.

\end{enumerate}

%\newpage
%
%\noindent
%I use the OSU Standard Grade Scheme: 
%
%\begin{center}
%\begin{tabular}{p{3cm}p{3cm}p{3cm}p{3cm}p{3cm}}
%93 - 100 (A)  & 83 - 86.9 (B) & 73 - 76.9 (C) & 60 - 66.9 (D) \\
%90 - 92.9 (A-) &  80 - 82.9 (B-) & 70 - 72.9 (C-) & Below 60 (E) \\ 
%87 - 89.9 (B+) & 77 - 79.9 (C+)  & 67 - 69.9 (D+) & \\ 
%\end{tabular}
%\end{center}


{\bf Academic misconduct:}
\\
I expect all the work you do in this course to be your own, unless
collaboration is explicitly requested for a particular task. Academic dishonesty will not be
allowed under any circumstances. Any case of cheating or plagiarism
will be handled according to academic policy.

{\bf Accommodations:}
\\
The University of Stuttgart strives to make all learning experiences as accessible as possible for students with disabilities or chronic illnesses. If you anticipate or experience academic barriers based on your disability or chronic illness, please let me know as soon as possible so that we can privately discuss options. For accommodations concerning exams please contact the chair of the Pr�fungsausschuss (\url{https://www.student.uni-stuttgart.de/beratung/pruefungsausschuss/}). Further information about accommodations and contact information for advisory services can be found here: \url{https://www.student.uni-stuttgart.de/en/counseling/disability/}.


%{\bf Special needs:}
%\\
%Students with disabilities that have been certified by the Office for
%Disability Services will be appropriately accommodated, and should
%inform the instructor as soon as possible of their needs. The Office
%for Disability Services is located in 150 Pomerene Hall, 1760 Neil
%Avenue; telephone 292-3307, TDD 292-0901;
%http://www.ods.ohio-state.edu/.


\bibliographystyle{/Users/tonhauser.1/Library/Latex/cslipubs-natbib}
\bibliography{/Users/tonhauser.1/Documents/bibliography}


\end{document}
