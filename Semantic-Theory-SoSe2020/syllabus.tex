
\documentclass[11pt,fleqn,a4]{article}
\usepackage{longtable}
\usepackage{enumitem}

\usepackage[margin=2cm]{geometry}
%\usepackage[dvips]{graphics}
\usepackage{hyperref}
\usepackage{txfonts}
\usepackage{amsfonts}
\usepackage{natbib}

\setlength{\parindent}{0in}
\setlength{\parskip}{1ex}
\setlength{\headsep}{0in}

\setlength{\bibsep}{0mm}
\bibpunct[:]{(}{)}{;}{a}{,}{,}

\newcommand{\yi}{\'{\symbol{16}}}
\newcommand{\nasi}{\~{\symbol{16}}}
\newcommand{\hina}{h\nasi na}
\newcommand{\ina}{\nasi na}
\newcommand{\tto}{t$_{top}$}
\newcommand{\wtop}{w$_{top}$}
\newcommand{\tc}{t$_c$}
\newcommand{\schwa}{\begin{sideways}e\end{sideways}}

% Semantic brackets
\newcommand{\6}{\mbox{$[\hspace*{-.6mm}[$}} 
\newcommand{\9}{\mbox{$]\hspace*{-.6mm}]$}}
\newcommand{\sem}[2]{\6#1\9$_{#2}$}
\newcommand{\jl}{\langle}
\newcommand{\jr}{\rangle}
\newcommand{\ibr}{-\hspace*{-0.1cm}i}
\newcommand{\hbr}{-\hspace*{-0.1cm}h}


\def\bad{{\leavevmode\llap{*}}}
\def\marginal{{\leavevmode\llap{?}}}
\def\verymarginal{{\leavevmode\llap{??}}}
\def\infelic{{\leavevmode\llap{\#}}}

\begin{document}

Sommersemester 2020 \hfill Linguistik/Anglistik, University of Stuttgart

\begin{center}

{\bf \large Semantic Theory (A and B)}
\\ Tuesdays 11:30-1pm, 2-3:30pm; online due to SARS-CoV-2

\end{center}

\vspace*{-.3cm}

{\bf Instructors:} 


\begin{itemize}

\item Dr.\ Judith Tonhauser 
\\ Office: 4.049 K2 
\\ Office hours: Fridays 1-2pm at \url{https://unistuttgart.webex.com/meet/judith.tonhauser}; for instructions, see \url{https://www.ling.uni-stuttgart.de/en/institute/team/Tonhauser/} 
\\ Email: {\em judith.tonhauser@ling.uni-stuttgart.de} 

\item Elena Vaik\v snorait\.{e}
\\ Office:  4.055 K2
\\ Office hours: Mondays 1-2pm at \url{https://meetingsemea3.webex.com/meet/elena.vaiksnoraite}
\\ Email: {\em elena.vaiksnoraite@ling.uni-stuttgart.de }

\end{itemize}

\medskip

Department of English linguistics front office: Ralf Bothner, 4.057 K2, Tel.\ +49 711 685-83120

{\bf Goals of the course:} The meanings conveyed by natural language utterances are subtle and complex, arising as a function of the conventional content of the expressions uttered and of various contextual factors. Semantics is the study of the conventional system, i.e., how meaning is coded in morphemes and complex expressions, while pragmatics studies how context plays a role in conveying meaning. This course is an introduction to the study of meaning, i.e., semantics and pragmatics. 

{\bf Readings:} The required readings for the course are chapters from Coppock \& Champollion's (2020) {\em Invitation to formal semantics} (please use the version uploaded to ILIAS, not the one linked to in the KVV) and Kroeger's  (2018) {\em Analyzing meaning}. Both books have been uploaded to ILIAS. \nocite{coppock-champollion,kroeger2018}

{\bf Course schedule:} 

`CC' refers to Coppock \& Champollion's {\em Invitation to formal semantics}
\\
`K' refers to Kroeger's {\em Analyzing meaning}

You are expected to have read and engaged with the required reading by the date for which it is listed.

\begin{longtable}{l l}
{\bf Reading due date} & {\bf Topic and required readings}  \\ \hline

21.04.20 & Screencast by JT and EV: Course overview, syllabus, how to succeed in this course   \\ 

28.04.20 & Entailment, conversational implicature, presupposition (CC ch.\ 1.1) \\

05.05.20 & Truth conditions, propositions, worlds, models (CC chs.\ 1.2.1, 1.2.2 \\

12.05.20 & Negative polarity items and set theory (CC chs.\ 2.1-2.3) \\

19.05.20 & Licensing of negative polarity items (CC ch.\ 2.4) \\

26.05.20 & Relations and functions (CC ch.\ 2.5) \\

02.06.20 & {\em No class, Pfingsferien} \\

09.06.20 & Propositional logic I (K chs.\ 4.1-4.2; CC chs. 3.1-3.2.2)\\

16.06.20 & Propositional logic II (CC chs.\ 3.2.3-3.3) \\

23.06.20 & Conversational implicatures, natural language versus propositional logic (K chs.\ 8.1-8.3, ch.\ 9) \\

30.06.20 & Aspect (K ch.\ 20, except 20.4.2) \\

07.07.20 & Tense (K ch.\ 21, except 21.4.2 and 21.5) \\ 

14.07.20 & Online final during class time (11:30am-1pm, 2-3:30pm) \\

\hline

\end{longtable}

{\bf How we will interact in this online version of the course:}

\begin{enumerate}[topsep=-3pt,itemsep=-1pt]

\item {\bf Weekly online quiz (on ILIAS):} These quizzes allow you to assess your understanding of the material. The quizzes are optional and not graded, though we will set them up in such a way that you can see whether your answers are correct. We strongly recommend that you take these quizzes so that you can get used to taking exams on ILIAS. Each quiz is available from 10am on the Thursday before the reading due date to noon on the reading due date (a Tuesday). 

\item {\bf Model answers:} We will post model answers for each quiz on ILIAS as soon as the quiz goes online. We highly recommend that you  first engage thoroughly with the reading and take the quiz before looking at the model answers. 

\item {\bf ILIAS fora:} There is a forum on ILIAS for each of the required readings. If you have a question about the reading, please post it there. We expect all of you to read these fora and answer your fellow classmates' questions as you are able to. 

\item {\bf Weekly feedback:} On Tuesday afternoon, JT and EV will give feedback on the week's material based on your performance on the quiz and your questions in the ILIAS forum. This feedback may be in the form of an email, a handout, or a screencast. We can, of course, only give feedback if you take the quiz and post questions on the forum! 

\item {\bf Study groups:} We recommend that you form study groups. The ILIAS course is set up in such a way that you can see who is enrolled.

\item {\bf Office hours:} Please come to our office hours to introduce yourselves and to ask about the course materials.

\end{enumerate}

%{\bf Important:} Each meeting will consist of me introducing new material and us doing exercises that will help you better understand the material covered. It is therefore vital that you attend class and regularly participate. If you have to miss class, please review the slides and to work with your class mates to understand what was covered.

\medskip

{\bf Assessment:} 
\\ Your grade for the course is determined by the online final exam on ILIAS, which will be available during the regular course hours during the last week of the semester (see the schedule above). The final is cumulative, open book and open notes. The term `open book' means that you can use the assigned readings; the term `open notes' means that you can use your notes. Note, however, that it is unlikely that you will need your notes or the books, if you have kept up with the material throughout the course. 

The final exam will be graded on the following scale: 

\setlength{\tabcolsep}{20pt}
\begin{tabular}{rrrr}
100-95,5 points: 1,0 & 87-83,5 points: 2,0 & 75-71,5 points: 3,0 & 63-60 points: 4,0  \\ 
95-91,5 points: 1,3 & 83-79,5 points: 2,3  & 71-67,5 points: 3,3 & 59-0 points: 5,0 \\
91-87,5 points: 1,7 & 79-75,5 points: 2,7 & 67-63,5 points: 3,7 & \\
\end{tabular}

\medskip

{\bf How to succeed in this course:}

\begin{itemize}[topsep=-3pt,itemsep=-1pt]

\item Keep up with the assigned readings! We have selected readings that you should be able to read on your own at this point in your study program. You should plan to spend around 4-6 hours on the reading each week. 

\item Take the quizzes! The quizzes will allow you to assess whether you have understood the assigned reading sufficiently well. You can take each quiz as often as you like. Since the final exam will also be implemented on ILIAS, taking the quizzes will also familiarize you with the platform.

\item Study the model answers! After you have engaged with the required readings and taken the quiz, you can compare your answers to the model answers.  If you have questions at this point, please post them on the ILIAS forum.

\item Form study groups! Discussing the required reading and quiz questions with others in the course will increase your understanding of the assigned material. 

\end{itemize}

\bigskip

%\vspace*{-.3cm}
%
%\begin{enumerate}[leftmargin=3ex]
%
%\item {\bf  Homework assignments} (ungraded VL): There will be three short ungraded homework assignments, to help prepare you for the final exam. You will upload your assignments to a folder on ILIAS and we will discuss the answers in class. 
%
%\item {\bf In-class final exam}: In the last class meeting, there will be a cumulative final exam (open book, open notes). The term `open book' means that you can bring the assigned readings; the term `open notes' means that you can bring your notes.
%
%\end{enumerate}

%\newpage
%
%\noindent
%I use the OSU Standard Grade Scheme: 
%
%\begin{center}
%\begin{tabular}{p{3cm}p{3cm}p{3cm}p{3cm}p{3cm}}
%93 - 100 (A)  & 83 - 86.9 (B) & 73 - 76.9 (C) & 60 - 66.9 (D) \\
%90 - 92.9 (A-) &  80 - 82.9 (B-) & 70 - 72.9 (C-) & Below 60 (E) \\ 
%87 - 89.9 (B+) & 77 - 79.9 (C+)  & 67 - 69.9 (D+) & \\ 
%\end{tabular}
%\end{center}


{\bf Academic misconduct:}
\\
We expect all the work you do in this course to be your own, unless
collaboration is explicitly requested for a particular task. Academic dishonesty will not be
allowed under any circumstances. Any case of cheating or plagiarism
will be handled according to academic policy.

{\bf Accommodations:}
\\
The University of Stuttgart strives to make all learning experiences accessible for students with disabilities or chronic illnesses. If you anticipate or experience academic barriers based on your disability or chronic illness, please let us know as soon as possible so that we can privately discuss options. For accommodations concerning exams please contact the chair of the Pr\"{u}fungsausschuss (\url{https://www.student.uni-stuttgart.de/beratung/pruefungsausschuss/}). Further information about accommodations and contact information for advisory services can be found here: \url{https://www.student.uni-stuttgart.de/en/counseling/disability/}.


%{\bf Special needs:}
%\\
%Students with disabilities that have been certified by the Office for
%Disability Services will be appropriately accommodated, and should
%inform the instructor as soon as possible of their needs. The Office
%for Disability Services is located in 150 Pomerene Hall, 1760 Neil
%Avenue; telephone 292-3307, TDD 292-0901;
%http://www.ods.ohio-state.edu/.


\bibliographystyle{cslipubs-natbib}
\bibliography{bibliography}


\end{document}
