
\documentclass[11pt,fleqn,a4]{article}
\usepackage{longtable}
\usepackage{enumitem}

\usepackage[margin=2cm]{geometry}
\usepackage[dvips]{graphics}
\usepackage{hyperref}
\usepackage{txfonts}
%\usepackage[spanish]{babel}
%\usepackage{linguex}
\usepackage{amsfonts}
\usepackage{natbib}
%\usepackage{gb4e}
%\usepackage[all]{xy}
%\usepackage{multirow}

\setlength{\parindent}{0in}
\setlength{\parskip}{1ex}
\setlength{\headsep}{0in}

\setlength{\bibsep}{0mm}
\bibpunct[:]{(}{)}{;}{a}{,}{,}

\newcommand{\yi}{\'{\symbol{16}}}
\newcommand{\nasi}{\~{\symbol{16}}}
\newcommand{\hina}{h\nasi na}
\newcommand{\ina}{\nasi na}
%\renewcommand{\abut}{$\supset$\hspace*{-0.07cm}$\subset$}
\newcommand{\tto}{t$_{top}$}
\newcommand{\wtop}{w$_{top}$}
\newcommand{\tc}{t$_c$}
\newcommand{\schwa}{\begin{sideways}e\end{sideways}}

% Semantic brackets
%\newcommand{\iss}[1]{\mbox{\protect\tiny \mbox{#1}}}
%\newcommand{\sem}[2]{\6#1\9$_\iss{#2}$} David's original
\newcommand{\6}{\mbox{$[\hspace*{-.6mm}[$}} 
\newcommand{\9}{\mbox{$]\hspace*{-.6mm}]$}}
\newcommand{\sem}[2]{\6#1\9$_{#2}$}
\newcommand{\jl}{\langle}
\newcommand{\jr}{\rangle}
\newcommand{\ibr}{-\hspace*{-0.1cm}i}
\newcommand{\hbr}{-\hspace*{-0.1cm}h}


\def\bad{{\leavevmode\llap{*}}}
\def\marginal{{\leavevmode\llap{?}}}
\def\verymarginal{{\leavevmode\llap{??}}}
\def\infelic{{\leavevmode\llap{\#}}}

\begin{document}

Sommersemester 2020 \hfill Linguistik/Anglistik, University of Stuttgart

\begin{center}

{\bf \large Semantic Theory (A and B)}
\\ Tuesdays; online due to SARS-CoV-2

\end{center}

\vspace*{-.3cm}

{\bf Instructors:} 


\begin{itemize}

\item Dr.\ Judith Tonhauser 
\\ Office: 4.049 K2 
\\ Office hours: Fridays 1-2pm at \url{https://unistuttgart.webex.com/meet/judith.tonhauser}; for instructions, see \url{https://www.ling.uni-stuttgart.de/en/institute/team/Tonhauser/} 
\\ Email: {\em judith.tonhauser@ling.uni-stuttgart.de} 

\item Elena 

\end{itemize}

\medskip

Department of English linguistics front office: Ralf Bothner, 4.057 K2, Tel.\ +49 711 685-83120

{\bf Goals of the course:} The meanings conveyed by natural language
utterances are subtle and complex, arising as a function of the
conventional content of the expressions uttered and of various contextual
factors. Semantics is the study of the conventional system itself, while
pragmatics studies how context plays a role in conveying meaning. This
course is an introduction to the study of meaning, i.e., semantics and
pragmatics. 

{\bf Readings:} The required reading for the course is Coppock \& Champollion's {\em Invitation to Formal Semantics} (please use the version uploaded to ILIAS) and {\bf WHAT FOR TENSE AND ASPECT?}.

{\bf Course schedule:} 

\begin{longtable}{l l}
{\bf Date} & {\bf Reading}  \\ \hline

21.04.20 & Course overview, syllabus   \\ 

28.04.20 & \\

05.05.20 & \\

12.05.20 & \\

19.05.20 & \\

26.05.20 & \\

02.06.20 & {\em No class, Pfingsferien} \\

09.06.20 & \\

16.06.20 & \\

23.06.20 & \\

30.06.20 & \\

07.07.20 & \\

14.07.20 & \\


& {\bf WHEN IS THE ONLINE EXAM?} \\
\hline

\end{longtable}

{\bf How we will interact in this online version of the course:}

\begin{enumerate}[topsep=-3pt,itemsep=-1pt]

\item Weekly readings screencast: These screencast will point out important or difficult concepts in the readings and relate the readings to the weekly exercise. The screencasts are uploaded to ILIAS on the Thursday before the reading due date (see course schedule).

\item Weekly online quiz (on ILIAS): These quizzes allow you to assess your understanding of the reading. The quizzes are optional and not graded, though we will set them up in such a way that you can see whether your answers are correct. We strongly recommend that you take these quizzes so that you can get used to taking exams on ILIAS. A subset of the quiz questions will be on the final exam. For each quiz, we will post model answers. The quiz will be available from 10am to 6pm on the chapter due date; if the Thursday is a holiday, they will be available at the same time on the immediately following Friday.

\item ILIAS fora: There is a forum on ILIAS for each of the chapters of the book that we will cover. If you have a question about the reading, please post it there. Any of you can answer these questions; I will answer them if nobody else has provided an answer, or if I feel the need to chime in.

\item Office hours: Come to my office hours to introduce yourself and ask questions about the course materials!

\item The ILIAS course is set up in such a way that you can see who is enrolled, so that you can form study groups.

\end{enumerate}

%{\bf Important:} Each meeting will consist of me introducing new material and us doing exercises that will help you better understand the material covered. It is therefore vital that you attend class and regularly participate. If you have to miss class, please review the slides and to work with your class mates to understand what was covered.


{\bf Assessment:} 
\\ Your grade for the course is determined by the online final exam, which will take place {\bf WHEN}. The final is cumulative, open book and open notes. The term `open book' means that you can bring the assigned readings; the term `open notes' means that you can bring your notes. 

The final exam will be graded on the following scale: 

\setlength{\tabcolsep}{20pt}
\begin{tabular}{rrrr}
100-95,5 points: 1,0 & 87-83,5 points: 2,0 & 75-71,5 points: 3,0 & 63-60 points: 4,0  \\ 
95-91,5 points: 1,3 & 83-79,5 points: 2,3  & 71-67,5 points: 3,3 & 59-0 points: 5,0 \\
91-87,5 points: 1,7 & 79-75,5 points: 2,7 & 67-63,5 points: 3,7 & \\
\end{tabular}

\medskip

%\vspace*{-.3cm}
%
%\begin{enumerate}[leftmargin=3ex]
%
%\item {\bf  Homework assignments} (ungraded VL): There will be three short ungraded homework assignments, to help prepare you for the final exam. You will upload your assignments to a folder on ILIAS and we will discuss the answers in class. 
%
%\item {\bf In-class final exam}: In the last class meeting, there will be a cumulative final exam (open book, open notes). The term `open book' means that you can bring the assigned readings; the term `open notes' means that you can bring your notes.
%
%\end{enumerate}

%\newpage
%
%\noindent
%I use the OSU Standard Grade Scheme: 
%
%\begin{center}
%\begin{tabular}{p{3cm}p{3cm}p{3cm}p{3cm}p{3cm}}
%93 - 100 (A)  & 83 - 86.9 (B) & 73 - 76.9 (C) & 60 - 66.9 (D) \\
%90 - 92.9 (A-) &  80 - 82.9 (B-) & 70 - 72.9 (C-) & Below 60 (E) \\ 
%87 - 89.9 (B+) & 77 - 79.9 (C+)  & 67 - 69.9 (D+) & \\ 
%\end{tabular}
%\end{center}


{\bf Academic misconduct:}
\\
I expect all the work you do in this course to be your own, unless
collaboration is explicitly requested for a particular task. Academic dishonesty will not be
allowed under any circumstances. Any case of cheating or plagiarism
will be handled according to academic policy.

{\bf Accommodations:}
\\
The University of Stuttgart strives to make all learning experiences accessible for students with disabilities or chronic illnesses. If you anticipate or experience academic barriers based on your disability or chronic illness, please let me know as soon as possible so that we can privately discuss options. For accommodations concerning exams please contact the chair of the Pr�fungsausschuss (\url{https://www.student.uni-stuttgart.de/beratung/pruefungsausschuss/}). Further information about accommodations and contact information for advisory services can be found here: \url{https://www.student.uni-stuttgart.de/en/counseling/disability/}.


%{\bf Special needs:}
%\\
%Students with disabilities that have been certified by the Office for
%Disability Services will be appropriately accommodated, and should
%inform the instructor as soon as possible of their needs. The Office
%for Disability Services is located in 150 Pomerene Hall, 1760 Neil
%Avenue; telephone 292-3307, TDD 292-0901;
%http://www.ods.ohio-state.edu/.


\bibliographystyle{/Users/tonhauser.1/Library/Latex/cslipubs-natbib}
\bibliography{/Users/tonhauser.1/Documents/bibliography}


\end{document}
